\chapter[Justificativa da abordagem]{Justificativa da abordagem}
  
    Com o passar do tempo, no mercado surgiram diversas abordagens para o desenvolvimento,
   as primeiras sendo chamadas tradicionais e, posteriormente, surgindo junto ao manifesto ágil,
   as adaptativas. Ambas abordagens possuem seus pontos fortes e fracos, dado esse fato veio a existir a necessidade
   de uma abordagem híbrida, de modo que quando necessário possam ser aproveitadas características de ambas as abordagens
   citadas, de acordo com Boehm e Turner (\citeyear{boehm}).
  
  \section{Metodologia para a análise da abordagem a se utilizar}
   
   A complexidade intrínseca ao desenvolvimento de software e a variedade de métodos disponíveis dificultam
   a comparação entre as abordagens tradicionais e as adaptativas, tornando-a, muitas vezes, imprecisa \cite{boehm}.
   Todavia, Boehm e Turner (\citeyear{boehm}) levantaram algumas características dos projetos de software, as quais possuem diferenças
   visíveis entre os métodos ágeis e tradicionais. São elas:
  
   \begin{itemize}
    \item \textbf{Características da aplicação} – incluem objetivos primários do projeto, tamanho do projeto e o ambiente da aplicação;
    \item \textbf{Características do gerenciamento} – incluem relações com o cliente, comunicação no projeto e planejamento e controle;
    \item \textbf{Características técnicas} – incluem abordagens à definição de requisitos, desenvolvimento e teste;
    \item \textbf{Características das pessoas} – incluem características do cliente, dos desenvolvedores e da cultura da organização.
   \end{itemize}   
   
   As observações feitas por Boehm e Turner (\citeyear{boehm}) se consolidaram no levantamento de campos de domínio (\textit{home ground}) específicos
   para os métodos tradicionais e adaptativos, e na identificação de cinco fatores críticos que podem ser utilizados para 
   determinar como um projeto se relaciona com esses \textit{home grounds}.
   
  \section{Proposta de abordagem}
    
    \subsection{Análise do contexto de negócio}
    
    \subsection{Análise da equipe}
    
    \subsection{Análise da interação com o cliente}
    
    \subsection{Considerações finais sobre a abordagem}