\chapter[Processo de Engenharia de Requisitos]{Processo de Engenharia de Requisitos}
  
  Como foi esclarecido no Capítulo 3, a abordagem a ser utilizada no projeto é uma abordagem híbrida do 
  \textit{Scaled Agile Framework} (SAFe) com o \textit{Rational Unified Process} (RUP).
  
  Para a definição do processo de Engenharia de Requisitos para o projeto, foi mantida a estrutura do SAFe com os seus três níveis,
  o nível de Portfólio, de Programa e de Time (como ilustra a Figura~\ref{safe_big_picture}), e trazidos alguns
  conceitos do RUP, como a utilização de casos de uso a nível de time. A estrutura dos três níveis do SAFe foi mantida porque,
  segundo Leffingwell (\citeyear{leffingwell11}), ao ir abaixando o nível de abstração dos requisitos mais gradativamente...
  
  \begin{figure}[!htbp]
    \centering
    \includegraphics[scale=0.13]{editaveis/figuras/SAFe_Big_Picture}
    \caption[The SAFe Big Picture]{\textit{The Big Picture}: \textit{Framework} proposto pelo SAFe\textregistered. \footnotemark}
    \label{safe_big_picture}
  \end{figure}
  \footnotetext{Fonte: Leffingwell, \textit{et al} (2008-2015). Disponível em <http://www.scaledagileframework.com/>}
  
  As próximas seções tratam sobre o modelo do processo de Engenharia de Requisitos, apelidado de '\textit{Big Picture}
  do projeto', os papéis desempenhados e as atividades e artefatos envolvidos no processo.
  
  \pagebreak
  \section{A '\textit{Big Picture}' do projeto}

        
  \begin{figure}[!htbp]
  \centering
  \includegraphics[scale=0.44, angle = 90]{editaveis/figuras/project_big_picture}
  \caption[Modelo do processo de Engenharia de Requisitos: \textit{Big Picture} do projeto]
      {Modelo do processo de Engenharia de Requisitos: \textit{Big Picture} do projeto.}
  \label{project_big_picture}
  \end{figure}

  \subsection{Subprocessos da '\textit{Big Picture}'}

    \begin{figure}[!htbp]
      \centering
      \includegraphics[scale=0.55]{editaveis/figuras/processo_identificar_epicos}
      \caption[Subprocesso - Identificar épicos de negócio]{Subprocesso - Identificar épicos de negócio.}
      \label{processo_identificar_epicos}
    \end{figure}

    \begin{figure}[!htbp]
      \centering
      \includegraphics[scale=0.55]{editaveis/figuras/processo_gerenciar_epicos}
      \caption[Subprocesso - Gerenciar épicos]{Subprocesso - Gerenciar épicos.}
      \label{processo_gerenciar_epicos}
    \end{figure}

    \begin{figure}[!htbp]
      \centering
      \includegraphics[scale=0.55]{editaveis/figuras/processo_levantar_features}
      \caption[Subprocesso - Levantar \textit{features}]{Subprocesso - Levantar \textit{features}.}
      \label{processo_levantar_features}
    \end{figure}

    \pagebreak
    \begin{figure}[!htbp]
      \centering
      \includegraphics[scale=0.55]{editaveis/figuras/processo_gerenciar_features}
      \caption[Subprocesso - Gerenciar \textit{features}]{Subprocesso - Gerenciar \textit{features}.}
      \label{processo_gerenciar_features}
    \end{figure}

    \begin{figure}[!htbp]
      \centering
      \includegraphics[scale=0.55]{editaveis/figuras/processo_especificar_casos_uso}
      \caption[Subprocesso - Especificar casos de uso]{Subprocesso - Especificar casos de uso.}
      \label{processo_especificar_casos_uso}
    \end{figure}

    \begin{figure}[!htbp]
      \centering
      \includegraphics[scale=0.55]{editaveis/figuras/processo_gerenciar_casos_uso}
      \caption[Subprocesso - Gerenciar casos de uso]{Subprocesso - Gerenciar casos de uso.}
      \label{processo_gerenciar_casos_uso}
    \end{figure}
    
  \pagebreak
  \section{Papéis no processo}
    
    
  Os papéis, e responsabilidades dos papéis, definidos para o processo são os seguintes:
  
  \begin{itemize}
   
   \item \textbf{\textit{Especialista do Negócio}}
   
      O Especialista do Negócio é o \textit{stakeholder} que detém o conhecimento do negócio, do contexto organizacional
      e da visão do produto.
      O professor atuará como Especialista do Negócio, majoritariamente, em todos os níveis (Portfólio, Programa e Time).
   
   \item \textbf{\textit{Product Owner} (PO)}
      
      Segundo conceitos apresentados por Leffingwell (\citeyear{leffingwell11}),
      são atividades de responsabilidade do PO, adaptados para o processo do projeto:
      
      \begin{itemize}
       
       \item Trabalhar com o \textit{Product Manager};
       
       \item Definir objetivos para a \textit{sprint};
       
       \item Priorizar e manter o \textit{Backlog} do Time;
       
       \item Elaborar e validar os casos de uso;
       
       \item Participar do planejamento da \textit{sprint} e validar a \textit{sprint}.
       
      \end{itemize}

      
      O professor atuará como PO para fornecer os insumos para as atividades que lhe couber, atuando como Especialista
      do Negócio, como as atividades de validação de \textit{sprints}, validação de casos de uso, entre outras do tipo.
      
      Fica para a equipe de requisitos a responsabilidade da execução das atividades, como planejamento de \textit{sprint},
      manutenção do \textit{Backlog}, entre outras. O papel do PO será exercido por membros da equipe de requisitos.
      
      O PO atuará no nível de Time.
      
   \item \textbf{\textit{Product Manager} (PM)}
      
      Segundo conceitos apresentados por Leffingwell (\citeyear{leffingwell11}),
      são atividades de responsabilidade do PM, adaptados para o processo do projeto:
      
      \begin{itemize}
       
       \item Manter a Visão e o \textit{Backlog do Programa};
       
       \item Priorizar \textit{features} e manter o \textit{Roadmap};
       
       \item Gerenciar o conteúdo da \textit{release};
       
       \item Manter e priorizar o \textit{Backlog do Portfólio};
       
      \end{itemize}
      
      O papel do PM será exercido por membros da equipe de requisitos, onde o professor atuará como Especialista de Negócio para
      o fornecimento das informações necessárias.
      
      O PM atuará nos níveis de programa e portfólio, atuando nas atividades relacionadas
      aos requisitos de mais alto nível.
      
      
   \item \textbf{\textit{Scrum Master}}
      
      O \textit{Scrum Master} é responsável por assistir o time para garantir a melhor perfomance, atuando com 
      um líder do time \cite{leffingwell11}.
      São responsabilidades do \textit{Scrum Master}, segundo Leffingwell (\citeyear{leffingwell11}):
      
      \begin{itemize}
       
       \item Facilitar o progresso do time;
       
       \item Liderar os esforços do time;
       
       \item Eliminar impedimentos;
       
      \end{itemize}
      
      O papel do \textit{Scrum Master} será realizado por um integrante da equipe de requisitos.
      
   \item \textbf{Time}
      
      O time é composto pelos desenvolvedores, que é representado por toda a equipe de requisitos.	
      
  \end{itemize}
  
  \vfill
    
  \pagebreak
  \section{Atividades e artefatos do processo}
    
    Nesta seção são abordadas os conceitos dos artefatos envolvidos no processo, bem como as atividades que se relacionam com
    esses artefatos.
    
      
  \subsection{Artefatos envolvidos}
  
    Descrever o conceito dos artefatos envolvidos...
    
  \subsection{Atividades e seus artefatos}
    
    
  \subsubsection{Nível de Portfólio}
    
    \begin{itemize}
     
     \item Atividade \textbf{Definir Tema de Investimento}
	
	\begin{itemize}
	  \item \textbf{Artefato(s) de entrada}: Nenhum.
	  
	  \item \textbf{Descrição}: Esta atividade consiste em definir o Tema de Investimento da organização,
	    para a derivação dos épicos.
	  
	  \item \textbf{Artefato(s) de saída}: Tema de Investimento definido.
	 	 
	\end{itemize}
	
     \item Subprocesso \textbf{Identificar os épicos de negócio}
	
	\begin{itemize}
	 \item Atividade \textbf{Analisar o negócio}
	     
	     \begin{itemize}
		\item \textbf{Artefato(s) de entrada}: Nenhum.
		
		\item \textbf{Descrição}: Esta atividade tem por objetivo entender o contexto do negócio, os problemas de mais alto nível
		  e estabelecer um vocabulário comum entre a equipe de desenvolvimento e o cliente (\textit{stakeholders}).
		
		\item \textbf{Artefato(s) de saída}: Lista de termos técnicos.
		  
	      \end{itemize}
	  
	  \item Atividade \textbf{Elicitar os épicos}
	  
	      \begin{itemize}
		\item \textbf{Artefato(s) de entrada}: Tema de Investimento.
		
		\item \textbf{Descrição}: Esta atividade tem por objetivo abstrair do Tema de Investimento definido as
		  iniciativas de desenvolvimento em larga escala (épicos), juntamente com os \textit{stakeholders},
		  a partir das técnicas de elicitação definidas.
		
		\item \textbf{Artefato(s) de saída}: Épicos (iniciais).
		      
	      \end{itemize}
	      
	  \item Atividade \textbf{Validar os épicos}
	  
	      \begin{itemize}
		\item \textbf{Artefato(s) de entrada}: Épicos (iniciais).
		
		\item \textbf{Descrição}: Esta atividade tem por objetivo validar os épicos que foram elicitados, juntamente
		com os \textit{stakeholders}. O \textit{Backlog} do Portfólio será composto dos épicos validados pelos
		\textit{stakeholders}.
		
		\item \textbf{Artefato(s) de saída}: \textit{Backlog} do Portfólio.
		      
	      \end{itemize}
	\end{itemize} % Fim das atividades de um subprocesso
	      
      \item Atividade \textbf{Priorizar épico}
      
	  \begin{itemize}
	    \item \textbf{Artefato(s) de entrada}: \textit{Backlog}  do Portfólio.
	    
	    \item \textbf{Descrição}: Esta atividade tem por objetivo definir o épico de maior relevância de onde serão
	      abstraídas as \textit{features}.
	    
	    \item \textbf{Artefato(s) de saída}: Épico priorizado.
		  
	  \end{itemize}
	   
	
     \item Subprocesso \textbf{Gerenciar épicos}
     
	\begin{itemize}
	 
	 \item Atividade \textbf{Atualizar \textit{Backlog} do Portfólio}
	    
	    \begin{itemize}
	      \item \textbf{Artefato(s) de entrada}: \textit{Backlog} do Portfólio.

	      \item \textbf{Descrição}: Consiste no acompanhamento do \textit{Backlog} do Portfólio, atualizando eventuais
		mudanças e épicos que já foram implementados e revisados.
	      
	      \item \textbf{Artefato(s) de saída}: \textit{Backlog} do Portfólio (atualizado).
		    
	    \end{itemize}
	    
	 \item Atividade \textbf{Verificar mudanças nos épicos}
	    
	    \begin{itemize}
	      \item \textbf{Artefato(s) de entrada}: \textit{Backlog} do Portfólio.

	      \item \textbf{Descrição}: Consiste em verificar se são necessárias mudanças nos épicos do
		\textit{Backlog} do Portfólio.
	      
	      \item \textbf{Artefato(s) de saída}: \textit{Backlog} do Portfólio (atualizado), se houver mudanças.
		    
	    \end{itemize}
	    
	 \item Atividade \textbf{Analisar impactos}
	    
	    \begin{itemize}
	      \item \textbf{Artefato(s) de entrada}: Matriz de rastreabilidade.

	      \item \textbf{Descrição}: Consiste em analisar os impactos causados por uma mudança em um ou
		mais épicos, e mitigar o problema, se houver.
	      
	      \item \textbf{Artefato(s) de saída}: Matriz de rastreabilidade (atualizada).
		    
	    \end{itemize}
	    
	\end{itemize}
     
    \end{itemize}
    
    \vfill
    
    \pagebreak
    
  \subsubsection{Nível de Programa}
    
    \begin{itemize}
	
     \item Subprocesso \textbf{Levantar as \textit{features}}
	
	\begin{itemize}
	  \item Atividade \textbf{Elicitar as \textit{features}}
	      
	      \begin{itemize}
		  \item \textbf{Artefato(s) de entrada}: Épico priorizado.
		  
		  \item \textbf{Descrição}: Esta atividade tem por objetivo abstrair do épico priorizado
		    as \textit{features} necessárias, juntamente com os \textit{stakeholders}, a partir das
		    técnicas de elicitação definidas.
		  
		  \item \textbf{Artefato(s) de saída}: \textit{Features} (iniciais).
		    
		\end{itemize}
	    
	    \item Atividade \textbf{Validar as \textit{features}}
	    
		\begin{itemize}
		  \item \textbf{Artefato(s) de entrada}: \textit{Features} (iniciais).
		  
		  \item \textbf{Descrição}: Esta atividade consiste em validar as \textit{features} elicitadas,
		    juntamente com os \textit{stakeholders}. O \textit{Backlog} do Programa será composto
		    pelas \textit{features} validadas pelos \textit{stakeholders}.
		  
		  \item \textbf{Artefato(s) de saída}: \textit{Backlog} do Programa.
			
		\end{itemize}
	\end{itemize} % Fim das atividades de um subprocesso
	      
     \item Atividade \textbf{Identificar os Requisitos não-funcionais}
      
	  \begin{itemize}
	    \item \textbf{Artefato(s) de entrada}: Nenhum.
	    
	    \item \textbf{Descrição}: Esta atividade tem por objetivo abstrair das necessidades do cliente os
	      requisitos não-funcionais do sistema. Se já houver requisitos não-funcionais identificados para o sistema,
	      os mesmos deverão ser consultados para atualizações.
	    
	    \item \textbf{Artefato(s) de saída}: Requisitos não-funcionais.
		  
	  \end{itemize}
	  
     \item Atividade \textbf{Elaborar \textit{Roadmap}}
      
	  \begin{itemize}
	    \item \textbf{Artefato(s) de entrada}: \textit{Backlog} do Programa.
	    
	    \item \textbf{Descrição}: Esta atividade tem por objetivo priorizar as \textit{features} do
	      \textit{Backlog} do Programa, planejando o arranjo das mesmas nas \textit{releases}.  Caso já exista
	      um \textit{Roadmap}, o mesmo deve ser atualizado com eventuais mudanças.
	    
	    \item \textbf{Artefato(s) de saída}: \textit{Roadmap}.
		  
	  \end{itemize}
	  
     \item Atividade \textbf{Definir Visão}
      
	  \begin{itemize}
	    \item \textbf{Artefato(s) de entrada}: Lista de termos técnicos, Requisitos não-funcionais e \textit{Roadmap}.
	    
	    \item \textbf{Descrição}: Esta atividade tem por objetivo elaborar (ou atualizar, caso já exista um
	      Documento de Visão) o Documento de Visão. O Documento de Visão deve ser validado com os \textit{stakeholders}.
	    
	    \item \textbf{Artefato(s) de saída}: Documento de Visão.
		  
	  \end{itemize}
	  
     \item Atividade \textbf{Priorizar uma \textit{feature}}
      
	  \begin{itemize}
	    \item \textbf{Artefato(s) de entrada}: \textit{Roadmap}.
	    
	    \item \textbf{Descrição}: Esta atividade consiste em definir a \textit{feature} de maior relevância
	      de onde serão derivados os casos de uso.
	    
	    \item \textbf{Artefato(s) de saída}: \textit{Feature} priorizada.
		  
	  \end{itemize}
	
     \item Subprocesso \textbf{Gerenciar \textit{features}}
     
	\begin{itemize}
	 
	 \item Atividade \textbf{Atualizar \textit{Backlog} do Programa}
	    
	    \begin{itemize}
	      \item \textbf{Artefato(s) de entrada}: \textit{Backlog} do Programa.

	      \item \textbf{Descrição}: Consiste no acompanhamento do \textit{Backlog} do Programa, atualizando
		eventuais mudanças e \textit{features} que já foram implementadas e revisadas.
	      
	      \item \textbf{Artefato(s) de saída}: \textit{Backlog} do Programa (atualizado).
		    
	    \end{itemize}
	    
	 \item Atividade \textbf{Verificar mudanças nas \textit{features}}
	    
	    \begin{itemize}
	      \item \textbf{Artefato(s) de entrada}: \textit{Backlog} do Programa.

	      \item \textbf{Descrição}: Consiste em verificar se são necessárias mudanças nas \textit{features} do
		\textit{Backlog} do Programa.
	      
	      \item \textbf{Artefato(s) de saída}: \textit{Backlog} do Programa (atualizado), se houver mudanças.
		    
	    \end{itemize}
	    
	 \item Atividade \textbf{Analisar impactos}
	    
	    \begin{itemize}
	      \item \textbf{Artefato(s) de entrada}: Matriz de rastreabilidade.

	      \item \textbf{Descrição}: Consiste em analisar os impactos causados por uma mudança em uma ou mais
		\textit{features}, e mitigar o problema, se houver.
	      
	      \item \textbf{Artefato(s) de saída}: Matriz de rastreabilidade (atualizada).
		    
	    \end{itemize}
	    
	\end{itemize}
	
     \item Atividade \textbf{Restrospectiva da \textit{release}}
      
	  \begin{itemize}
	    \item \textbf{Artefato(s) de entrada}: \textit{Backlog} do Time e
	      Incremento de \textit{software} (\textit{Build} produzida pelo time na \textit{release}).
	    
	    \item \textbf{Descrição}: Esta atividade tem por objetivo validar o que foi produzido na \textit{release},
	      juntamente com os envolvidos.
	    
	    \item \textbf{Artefato(s) de saída}: Resumo da retrospectiva da \textit{release}.
		  
	  \end{itemize}
     
    \end{itemize}
    

    
    
	