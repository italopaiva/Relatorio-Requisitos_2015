
  Os papéis, e responsabilidades dos papéis, definidos para o processo são os seguintes:
  
  \begin{itemize}
   
   \item \textbf{\textit{Especialista do Negócio}}
   
      O Especialista do Negócio é o \textit{stakeholder} que detém o conhecimento do negócio, do contexto organizacional
      e da visão do produto.
      O professor atuará como Especialista do Negócio, majoritariamente, em todos os níveis (Portfólio, Programa e Time).
   
   \item \textbf{\textit{Product Owner} (PO)}
      
      Segundo conceitos apresentados por Leffingwell (\citeyear{leffingwell11}),
      são atividades de responsabilidade do PO, adaptados para o processo do projeto:
      
      \begin{itemize}
       
       \item Trabalhar com o \textit{Product Manager};
       
       \item Definir objetivos para a \textit{sprint};
       
       \item Priorizar e manter o \textit{Backlog} do Time;
       
       \item Elaborar e validar os casos de uso;
       
       \item Participar do planejamento da \textit{sprint} e validar a \textit{sprint}.
       
      \end{itemize}

      
      O professor atuará como PO para fornecer os insumos para as atividades que lhe couber, atuando como Especialista
      do Negócio, como as atividades de validação de \textit{sprints}, validação de casos de uso, entre outras do tipo.
      
      Fica para a equipe de requisitos a responsabilidade da execução das atividades, como planejamento de \textit{sprint},
      manutenção do \textit{Backlog}, entre outras. O papel do PO será exercido por membros da equipe de requisitos.
      
      O PO atuará no nível de Time.
      
   \item \textbf{\textit{Product Manager} (PM)}
      
      Segundo conceitos apresentados por Leffingwell (\citeyear{leffingwell11}),
      são atividades de responsabilidade do PM, adaptados para o processo do projeto:
      
      \begin{itemize}
       
       \item Manter a Visão e o \textit{Backlog do Programa};
       
       \item Priorizar \textit{features} e manter o \textit{Roadmap};
       
       \item Gerenciar o conteúdo da \textit{release};
       
       \item Manter e priorizar o \textit{Backlog do Portfólio};
       
      \end{itemize}
      
      O papel do PM será exercido por membros da equipe de requisitos, onde o professor atuará como Especialista de Negócio para
      o fornecimento das informações necessárias.
      
      O PM atuará nos níveis de programa e portfólio, atuando nas atividades relacionadas
      aos requisitos de mais alto nível.
      
      
   \item \textbf{\textit{Scrum Master}}
      
      O \textit{Scrum Master} é responsável por assistir o time para garantir a melhor perfomance, atuando com 
      um líder do time \cite{leffingwell11}.
      São responsabilidades do \textit{Scrum Master}, segundo Leffingwell (\citeyear{leffingwell11}):
      
      \begin{itemize}
       
       \item Facilitar o progresso do time;
       
       \item Liderar os esforços do time;
       
       \item Eliminar impedimentos;
       
      \end{itemize}
      
      O papel do \textit{Scrum Master} será realizado por um integrante da equipe de requisitos.
      
   \item \textbf{Time}
      
      O time é composto pelos desenvolvedores, que é representado por toda a equipe de requisitos.	
      
  \end{itemize}
  
  \vfill