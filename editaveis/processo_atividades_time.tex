
  \subsubsection{Nível de Time}
    
    \begin{itemize}
	
     \item Subprocesso \textbf{Especificar Casos de Uso}
	
	\begin{itemize}
	  \item Atividade \textbf{Elicitar casos de uso}
	      
	      \begin{itemize}
		  \item \textbf{Artefato(s) de entrada}: \textit{Feature} priorizada.
		  
		  \item \textbf{Descrição}: Consiste em levantar os casos de uso juntamente com os clientes
		    (\textit{stakeholders}), utilizando as técnicas de elicitação definidas,  a partir de determinada
		    \textit{feature}.
		  
		  \item \textbf{Artefato(s) de saída}: Casos de uso (iniciais).
		    
		\end{itemize}
	    
	    \item Atividade \textbf{Validar casos de uso}.
	    
		\begin{itemize}
		  \item \textbf{Artefato(s) de entrada}: Casos de uso (iniciais).
		  
		  \item \textbf{Descrição}: Consiste em validar os casos de uso elicitados, juntamente com os
		    clientes (\textit{stakeholders}).
		  
		  \item \textbf{Artefato(s) de saída}: \textit{Backlog} do Time com os casos de uso que foram validados.
			
		\end{itemize}
		
	    \item Atividade \textbf{Escrever a descrição dos casos de uso}.
	    
		\begin{itemize}
		  \item \textbf{Artefato(s) de entrada}: \textit{Backlog} do Time.
		  
		  \item \textbf{Descrição}: Consiste em descrever detalhadamente os casos de uso do \textit{Backlog} do time.
		  
		  \item \textbf{Artefato(s) de saída}: Descrição dos casos de uso.
			
		\end{itemize}
		
	    \item Atividade \textbf{Elaborar diagrama de casos de uso}.
	    
		\begin{itemize}
		  \item \textbf{Artefato(s) de entrada}: Casos de uso (iniciais).
		  
		  \item \textbf{Descrição}: Consiste em modelar (UML) os casos de uso do \textit{Backlog} do time.
		  
		  \item \textbf{Artefato(s) de saída}: Diagrama de casos de uso.
			
		\end{itemize}
		
	    \item Atividade \textbf{Elaborar modelos de casos de uso}.
	    
		\begin{itemize}
		  \item \textbf{Artefato(s) de entrada}: Descrição de casos de uso e Diagrama de casos de uso.
		  
		  \item \textbf{Descrição}: Consiste em construir o modelo de casos de uso. O modelo de casos de uso
		   é construído incrementalmente.
		  
		  \item \textbf{Artefato(s) de saída}: Modelos de casos de uso.
			
		\end{itemize}
	\end{itemize} % Fim das atividades de um subprocesso
	      
     \item Atividade \textbf{Priorizar casos de uso}.
      
	  \begin{itemize}
	    \item \textbf{Artefato(s) de entrada}: \textit{Backlog} do Time.
	    
	    \item \textbf{Descrição}: Consiste em priorizar os casos de uso do \textit{Backlog} do Time para
	      encaixá-los em uma \textit{release} (conjunto de \textit{sprints}).
	    
	    \item \textbf{Artefato(s) de saída}: \textit{Backlog} do Time (priorizado).
		  
	  \end{itemize}
	  
     \item Atividade \textbf{Planejar \textit{sprint}}
      
	  \begin{itemize}
	    \item \textbf{Artefato(s) de entrada}: \textit{Backlog} do Programa.
	    
	    \item \textbf{Descrição}: Consiste em selecionar os casos de uso do \textit{Backlog} do time que serão
	      implementados na \textit{sprint}, criando o \textit{Backlog} da \textit{sprint}, e dividir as
	      responsabilidades de desenvolvimento para os integrantes do time.
	    
	    \item \textbf{Artefato(s) de saída}: \textit{Backlog} da \textit{sprint}.
		  
	  \end{itemize}
	  
     \item Atividade \textbf{Executar \textit{sprint}}
      
	  \begin{itemize}
	    \item \textbf{Artefato(s) de entrada}: \textit{Backlog} da \textit{sprint}.
	    
	    \item \textbf{Descrição}: Consiste em implementar os casos de uso do \textit{Backlog} da \textit{sprint}
	      e validar o que foi feito a cada término com o time, além das reuniões diárias (\textit{stand-up meeting}).
	    
	    \item \textbf{Artefato(s) de saída}: Incremento de \textit{software} (\textit{Build}).
		  
	  \end{itemize}
	  
     \item Atividade \textbf{Fazer restrospectiva da \textit{sprint}}
      
	  \begin{itemize}
	    \item \textbf{Artefato(s) de entrada}: \textit{Backlog} da \textit{sprint}.
	    
	    \item \textbf{Descrição}: Consiste numa reunião do time para avaliar o que foi feito e analisar pontos
	      positivos, pontos negativos e possíveis melhorias para a próxima \textit{sprint}.
	    
	    \item \textbf{Artefato(s) de saída}: Resumo da retrospectiva, com os pontos levantados pela equipe
	     (possível \textit{KanBan} de retrospectiva).
		  
	  \end{itemize}
	  
     \item Atividade \textbf{Fazer revisão da \textit{sprint}}
      
	  \begin{itemize}
	    \item \textbf{Artefato(s) de entrada}: \textit{Backlog} da \textit{sprint}, \textit{Backlog} do time,
	      Incremento de \textit{software} gerado na \textit{sprint} (\textit{Build}).
	    
	    \item \textbf{Descrição}: Consiste numa reunião com os envolvidos para avaliar o que foi gerado na \textit{sprint}.
	      Permite alinhar a visão do time com a dos \textit{stakeholders}, verificando se o que foi feito está
	      alinhado com os propósitos iniciais e se haverá mudanças no que será feita em seguida (\textit{Backlog} do time).
	      	    
	    \item \textbf{Artefato(s) de saída}: Resumo da revisão da \textit{sprint}.
		  
	  \end{itemize}
	  
      \item Atividade \textbf{Gerenciar a \textit{sprint}}
      
	  \begin{itemize}
	    \item \textbf{Artefato(s) de entrada}: \textit{Backlog} da \textit{sprint}.
	    
	    \item \textbf{Descrição}: Consiste no acompanhamento e suporte do time durante a \textit{sprint},
	      gerindo o desenvolvimento e possíveis mudanças.
	    
	    \item \textbf{Artefato(s) de saída}: Nenhum.
		  
	  \end{itemize}
	
     \item Subprocesso \textbf{Gerenciar casos de uso}
     
	\begin{itemize}
	 
	 \item Atividade \textbf{Atualizar \textit{Backlog} do Time}
	    
	    \begin{itemize}
	      \item \textbf{Artefato(s) de entrada}: \textit{Backlog} do Time.

	      \item \textbf{Descrição}: Consiste no acompanhamento do \textit{Backlog} do Time, atualizando eventuais
		mudanças e casos de uso que já foram implementados e revisados.
	      
	      \item \textbf{Artefato(s) de saída}: \textit{Backlog} do Time (atualizado).
		    
	    \end{itemize}
	    
	 \item Atividade \textbf{Verificar mudanças nos casos de uso}
	    
	    \begin{itemize}
	      \item \textbf{Artefato(s) de entrada}: \textit{Backlog} do Time e Resumo da revisão de \textit{sprint}.

	      \item \textbf{Descrição}: Consiste em verificar se são necessárias mudanças nos casos de uso do \textit{Backlog}
		do Time. Pode acontecer mudanças dependendo do resultado da revisão de \textit{sprint}.
	      
	      \item \textbf{Artefato(s) de saída}: \textit{Backlog} do Time (atualizado).
		    
	    \end{itemize}
	    
	 \item Atividade \textbf{Analisar impactos}
	    
	    \begin{itemize}
	      \item \textbf{Artefato(s) de entrada}: Matriz de rastreabilidade.

	      \item \textbf{Descrição}: Consiste em analisar os impactos causados por uma mudança em um ou mais casos de uso
	       em outros casos de uso, e mitigar o problema, se houver.
	      
	      \item \textbf{Artefato(s) de saída}: Matriz de rastreabilidade (atualizada).
		    
	    \end{itemize}
	    
	\end{itemize}
	     
    \end{itemize}