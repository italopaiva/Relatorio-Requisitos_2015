
  \subsubsection{Nível de Programa}
    
    \begin{itemize}
	
     \item Subprocesso \textbf{Levantar as \textit{features}}
	
	\begin{itemize}
	  \item Atividade \textbf{Elicitar as \textit{features}}
	      
	      \begin{itemize}
		  \item \textbf{Artefato(s) de entrada}: Épico priorizado.
		  
		  \item \textbf{Descrição}: Esta atividade tem por objetivo abstrair do épico priorizado
		    as \textit{features} necessárias, juntamente com os \textit{stakeholders}, a partir das
		    técnicas de elicitação definidas.
		  
		  \item \textbf{Artefato(s) de saída}: \textit{Features} (iniciais).
		    
		\end{itemize}
	    
	    \item Atividade \textbf{Validar as \textit{features}}
	    
		\begin{itemize}
		  \item \textbf{Artefato(s) de entrada}: \textit{Features} (iniciais).
		  
		  \item \textbf{Descrição}: Esta atividade consiste em validar as \textit{features} elicitadas,
		    juntamente com os \textit{stakeholders}. O \textit{Backlog} do Programa será composto
		    pelas \textit{features} validadas pelos \textit{stakeholders}.
		  
		  \item \textbf{Artefato(s) de saída}: \textit{Backlog} do Programa.
			
		\end{itemize}
	\end{itemize} % Fim das atividades de um subprocesso
	      
     \item Atividade \textbf{Identificar os Requisitos não-funcionais}
      
	  \begin{itemize}
	    \item \textbf{Artefato(s) de entrada}: \textit{Backlog} do Programa.
	    
	    \item \textbf{Descrição}: Esta atividade tem por objetivo abstrair das necessidades do cliente os
	      requisitos não-funcionais do sistema. Se já houver requisitos não-funcionais identificados para o sistema,
	      os mesmos deverão ser consultados para atualizações.
	    
	    \item \textbf{Artefato(s) de saída}: Requisitos não-funcionais.
		  
	  \end{itemize}
	  
     \item Atividade \textbf{Elaborar \textit{Roadmap}}
      
	  \begin{itemize}
	    \item \textbf{Artefato(s) de entrada}: \textit{Backlog} do Programa.
	    
	    \item \textbf{Descrição}: Esta atividade tem por objetivo priorizar as \textit{features} do
	      \textit{Backlog} do Programa, planejando o arranjo das mesmas nas \textit{releases}.  Caso já exista
	      um \textit{Roadmap}, o mesmo deve ser atualizado com eventuais mudanças.
	    
	    \item \textbf{Artefato(s) de saída}: \textit{Roadmap}.
		  
	  \end{itemize}
	  
     \item Atividade \textbf{Definir Visão}
      
	  \begin{itemize}
	    \item \textbf{Artefato(s) de entrada}: Lista de termos técnicos, Requisitos não-funcionais e \textit{Roadmap}.
	    
	    \item \textbf{Descrição}: Esta atividade tem por objetivo elaborar (ou atualizar, caso já exista um
	      Documento de Visão) o Documento de Visão. O Documento de Visão deve ser validado com os \textit{stakeholders}.
	    
	    \item \textbf{Artefato(s) de saída}: Documento de Visão.
		  
	  \end{itemize}
	  
     \item Atividade \textbf{Priorizar \textit{features}}
      
	  \begin{itemize}
	    \item \textbf{Artefato(s) de entrada}: \textit{Roadmap}.
	    
	    \item \textbf{Descrição}: Esta atividade consiste em definir as \textit{features} de maior relevância
	      de onde serão derivados os casos de uso.
	    
	    \item \textbf{Artefato(s) de saída}: \textit{Features} priorizadas.
		  
	  \end{itemize}
	
     \item Subprocesso \textbf{Gerenciar \textit{features}}
     
	\begin{itemize}
	 
	 \item Atividade \textbf{Atualizar \textit{Backlog} do Programa}
	    
	    \begin{itemize}
	      \item \textbf{Artefato(s) de entrada}: \textit{Backlog} do Programa.

	      \item \textbf{Descrição}: Consiste no acompanhamento do \textit{Backlog} do Programa, atualizando
		eventuais mudanças e \textit{features} que já foram implementadas e revisadas.
	      
	      \item \textbf{Artefato(s) de saída}: \textit{Backlog} do Programa (atualizado).
		    
	    \end{itemize}
	    
	 \item Atividade \textbf{Verificar mudanças nas \textit{features}}
	    
	    \begin{itemize}
	      \item \textbf{Artefato(s) de entrada}: \textit{Backlog} do Programa.

	      \item \textbf{Descrição}: Consiste em verificar se são necessárias mudanças nas \textit{features} do
		\textit{Backlog} do Programa.
	      
	      \item \textbf{Artefato(s) de saída}: \textit{Backlog} do Programa (atualizado), se houver mudanças.
		    
	    \end{itemize}
	    
	 \item Atividade \textbf{Analisar impactos}
	    
	    \begin{itemize}
	      \item \textbf{Artefato(s) de entrada}: Matriz de rastreabilidade.

	      \item \textbf{Descrição}: Consiste em analisar os impactos causados por uma mudança em uma ou mais
		\textit{features}, e mitigar o problema, se houver.
	      
	      \item \textbf{Artefato(s) de saída}: Matriz de rastreabilidade (atualizada).
		    
	    \end{itemize}
	    
	\end{itemize}
	
     \item Atividade \textbf{Restrospectiva da \textit{release}}
      
	  \begin{itemize}
	    \item \textbf{Artefato(s) de entrada}: \textit{Backlog} do Time e 
	      \textit{Build} produzida pelo time na \textit{release}.
	    
	    \item \textbf{Descrição}: Esta atividade tem por objetivo validar o que foi produzido na \textit{release},
	      juntamente com os envolvidos.
	    
	    \item \textbf{Artefato(s) de saída}: Resumo da retrospectiva da \textit{release}.
		  
	  \end{itemize}
     
    \end{itemize}
    
    \vfill