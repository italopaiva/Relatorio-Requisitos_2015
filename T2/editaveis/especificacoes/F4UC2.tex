
  \section{Caso de Uso F4UC2 - Consultar viatura}

  {\raggedright
      \textbf{Descrição}
  }

Este caso de uso permite que o Gestor Administrativo consulte a situação das viaturas do Corpo de 
Bombeiros do Distrito Federal (CBMDF).
    
  {\raggedright
      \textbf{Ator Principal}
  }

    Gestor Administrativo. O Gestor Administrativo representa os Tenentes e Capitães da corporação do Corpo de Bombeiros do Distrito
    Federal (CBMDF).

  {\raggedright
      \textbf{Fluxo Principal}
  }
  
Este caso de uso é iniciado quando o Gestor Administrativo escolhe a opção de consultar viatura.

  \begin{enumerate}

 \item O sistema solicita ao Gestor Administrativo o preenchimento dos dados;
 \item O Gestor Administrativo fornece todos os dados e solicita a consulta; [RN02][FE01]
 \item O sistema apresenta os dados obtidos na consulta. [RN01][RN03]
    
  \end{enumerate}
  
  
   {\raggedright
      \textbf{Regras de Negócio}
   }
   
   \textbf{RN01} – Atributos de Viatura.
   
 A tabela a seguir apresenta os atributos que deverão estar presentes em todas as viaturas.
 
   \begin{table*}[!h]
    \centering
      \begin{tabular}{|p{0.20\linewidth}|p{0.4\linewidth}|p{0.20\linewidth}|p{0.20\linewidth}|}
      \hline
      Campo  & Formato & Valores & Obrigatoriedade\\
      \hline

        Número de identificação & AAA000 & - & Sim\\

  \hline                               
  Unidade de operação & - & - & Sim\\

  \hline                               
  Status & - & Em manutenção, ativa e desativada & Sim\\
  
  \hline                               
  Tipo de viatura & - & - & Sim\\
  
  \hline                               
  Tipo de combustível & - & Álcool, Diesel e Gasolina & Sim\\
  
  \hline                               
  Marca do veículo & - & - & Sim\\
  
  \hline                               
  Modelo do veículo & - & - & Sim\\
  
  \hline                               
  Placa & AAA-0000 & - & Não\\
  
  \hline                               
  Chassi & AA.AA.AAAAA.A.A.AAAAAA & - & Não\\
  
  \hline                               
  Responsável & - & - & Sim\\  
  
  \hline  
      
      \end{tabular}
    \end{table*}

  \vfill
  \pagebreak
    
      \textbf{RN02} – Parâmetros de Pesquisa.
   
A tabela a seguir apresenta os parâmetros utilizados para a pesquisa.

   \begin{table*}[!h]
    \centering
      \begin{tabular}{|p{0.20\linewidth}|p{0.25\linewidth}|p{0.25\linewidth}|p{0.25\linewidth}|}
      \hline
      Campo & Formato & Valores & Obrigatoriedade\\
    
  \hline                               
  Número de identificação & AAA000 & - & Sim\\

  \hline                               
  Status & - & Em manutenção, ativa e desativada & Sim\\
  
  \hline                               
  Tipo de viatura & - & - & Sim\\
  
  \hline     
        
      \end{tabular}
    \end{table*}
    
  \textbf{RN03} – Tipos de viatura.
   
 A tabela a seguir apresenta os tipos de veículos presentes no CBMDF e os valores que podem assumir.
 
   \begin{table*}[!h]
    \centering
      \begin{tabular}{|p{0.20\linewidth}|p{0.25\linewidth}|}
      \hline
      Tipo & Valores\\
    
  \hline                               
  Terreste & Caminhão, carro de passeio e motocicleta\\

  \hline                               
  Aquático & Barco e Jetski\\

  \hline                               
  Aéreo & Helicóptero\\
  
  \hline     
        
      \end{tabular}
    \end{table*}

    
    
   {\raggedright
      \textbf{Fluxo de Exceção}
   }
   
   FE01. Validação de dados
	 No passo 2 do fluxo principal, se um ou mais campos não forem validados (no formato e/ou obrigatoriedade de 
 preenchimento) o sistema exibe uma mensagem informando o ocorrido ao Gestor Administrativo. Retorna ao 
 primeiro passo do fluxo principal.


 %\vfill
 %\pagebreak

	
   {\raggedright
      \textbf{Condições}
   }
   
    
   \textbf{Pré-condições}
   
   O Gestor Administrativo deve estar corretamente autenticado no sistema para a execução do caso de uso.
   
   \textbf{Pós-condição}
   
   A operação realizada no caso de uso deve ser registrada, juntamente com o autor, data e horário da operação, para fins de auditorias futuras.
  
  \vfill
  \pagebreak


   
 
