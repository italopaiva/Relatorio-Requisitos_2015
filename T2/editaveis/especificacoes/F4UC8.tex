
  \section{Caso de Uso F4UC8 - Inativar motorista}

  {\raggedright
      \textbf{Descrição}
  }

    Este caso de uso permite que o Gestor Administrativo inative  um motorista cadastrado na corporação do Corpo de Bombeiros do
    Distrito Federal (CBMDF). Um motorista nunca poderá ter seu registro excluído. O mesmo apenas terá seu status alterado e será
    inativado.
    
  {\raggedright
      \textbf{Ator Principal}
  }

    Gestor Administrativo. O Gestor Administrativo representa os Tenentes e Capitães da corporação dos Corpo de Bombeiros do
    Distrito Federal (CBMDF).


  {\raggedright
      \textbf{Fluxo Principal}
  }
  
    Este caso de uso é iniciado quando o Gestor Administrativo escolhe a opção de Inativar um motorista.
  
  \begin{enumerate}
    \item O sistema solicita ao Gestor Administrativo que informe o motorista que deseja-se inativar.
    \item O Gestor Administrativo  informa os dados do motorista pelos quais deseja fazer a pesquisa e solicita a consulta.[PE01]
    \item O sistema apresenta os dados do motorista solicitado.[FE01]
    \item O Gestor Administrativo solicita a inativação.
    \item O sistema exibe uma mensagem de alerta ao Gestor Administrativo perguntando se, realmente, o mesmo deseja inativar o
    motorista solicitado.
    \item O Gestor Administrativo confirma a inativação do motorista.
    \item O sistema inativa o motorista.[RN01]
    \item O sistema apresenta uma mensagem de confirmação da inativação.
    \item O caso de uso é encerrado.

  \end{enumerate}
  
  
   {\raggedright
      \textbf{Regras de Negócio}
   }
   
   \textbf{RN01} – Inativação do motorista
   
    O Registro do motorista tem seu status alterado para “Inativo”.
    
    \vfill
    \pagebreak
    
   {\raggedright
      \textbf{Fluxo de Exceção}
   }
	
    FE01. Motorista não encontrado
	No passo 3 do fluxo principal, o sistema não encontra o motorista informado pelo Gestor Administrativo. O sistema exibe uma
	mensagem de erro ao Gestor Administrativo e retorna ao passo 2 do fluxo principal.

	
   {\raggedright
      \textbf{Condições}
   }
   
    
   \textbf{Pré-condições}
   
   O Gestor Administrativo deve estar corretamente autenticado no sistema para a execução do caso de uso.
   
   \textbf{Pós-condição}
   
   A operação realizada no caso de uso deve ser registrada, juntamente com o autor, data e horário da operação, para fins de auditorias futuras.


   {\raggedright
      \textbf{Ponto de Extensão}
   }
   
   
    PE01. Incluir o caso de uso: Consultar motorista.
	    No passo 2 do fluxo principal, o caso de uso Consultar motorista deve ser executado.
   


  \vfill
  \pagebreak