
  \section{Caso de Uso F4UC9 - Gerenciar vínculo entre motoristas e viaturas}

  {\raggedright
      \textbf{Descrição}
  }

   Este caso de uso permite a um Gestor Administrativo ou a um Sargento de uma unidade, vincular um motorista à uma viatura e
   desvincular um motorista de uma viatura da unidade para poder deixar pré-associado a uma viatura seus respectivos motoristas.

    
  {\raggedright
      \textbf{Ator Principal}
  }

     Gestor administrativo. O Gestor administrativo representa os tenentes e capitães alocados em uma unidade do Corpo de Bombeiros Militar do Distrito Federal (CBMDF).

  {\raggedright
      \textbf{Fluxo Principal}
  }
  
    Este caso de uso é iniciado quando o Gestor administrativo escolhe a opção que permite gerenciar vínculos entre motorista e viaturas.
    
  
  \begin{enumerate}
    \item O sistema exibe as viaturas da unidade.
    \item O Gestor administrativo seleciona uma viatura.
    \item Se houver motoristas associados à viatura, o sistema exibe-os ao Gestor administrativo.[FA01]
    \item O Gestor administrativo seleciona a opção que permite vincular motoristas a viatura.
    \item O sistema exibe os motoristas da unidade que estão aptos para serem vinculados à viatura selecionada pelo Gestor administrativo.[RN01][RN02][FE01]
    \item O Gestor administrativo vincula o motorista desejado à viatura.
    \item O sistema apresenta uma mensagem de confirmação da vinculação do motorista à viatura.
    \item O caso de uso é encerrado.
  \end{enumerate}
  
  {\raggedright
      \textbf{Fluxo Alternativo}
  }
  
    \textbf{FA01} - Desvincular motoristas da viatura
  
    No passo 3 do fluxo principal, esse fluxo alternativo é iniciado quando o Gestor administrativo seleciona um motorista que está
    associado à viatura e seleciona a opção que permite desvincular um motorista da viatura.

    \begin{enumerate}
      \item O Gestor administrativo seleciona um motorista associado à viatura.
      \item O Gestor administrativo desvincula o motorista da viatura.	
      \item O sistema desassocia o motorista da viatura.
      \item O sistema informa uma mensagem confirmando que o motorista foi desvinculado da viatura.
      \item O fluxo alternativo é encerrado.
      
    \end{enumerate}
  
  
   {\raggedright
      \textbf{Regras de Negócio}
   }
   
   \textbf{RN01} - Os motoristas devem cumprir os seguintes requisitos para poderem dirigir um tipo de viatura:
   
    \begin{itemize}
     \item Possuir habilitação para o tipo de viatura.
     \item Possuir o curso específico para o tipo de viatura.
    \end{itemize}

   \textbf{RN02} - Um motorista só pode ser vinculado a uma viatura
   
    Um motorista que já está vinculado a uma viatura não está apto para ser vinculado a outra viatura.

   
   {\raggedright
      \textbf{Fluxo de Exceção}
   }
   
    FE01 - Não há motoristas aptos para o tipo de viatura escolhido.

      No passo 3 do fluxo principal, o sistema disponibiliza uma lista dos motoristas aptos para o tipo de viatura selecionada.
      Caso não haja nenhum motorista apto, o sistema deve informar uma mensagem explicitando o ocorrido. Após a mensagem, o sistema
      volta ao passo 1 do fluxo principal para permitir a escolha de outra viatura.
	
   {\raggedright
      \textbf{Condições}
   }
   
    
   \textbf{Pré-condições}
   
   O Gestor administrativo devem estar corretamente autenticados no sistema para a execução do caso de uso.
   
   \textbf{Pós-condição}
   
   A operação realizada no caso de uso deve ser registrada, juntamente com o autor, data e horário da operação, para fins de auditorias futuras.

  \vfill
  \pagebreak