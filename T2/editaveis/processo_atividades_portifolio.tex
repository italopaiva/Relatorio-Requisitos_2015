
  \subsubsection{Nível de Portfólio}
    
    \begin{itemize}
      
      \item Atividade \textbf{Analisar o negócio}
	  
	  \begin{itemize}
	    \item \textbf{Artefato(s) de entrada}: Contexto de negócio, Processo atual de negócio.
	    
	    \item \textbf{Descrição}: Esta atividade tem por objetivo entender o contexto do negócio, os problemas de mais alto nível
	      e estabelecer um vocabulário comum entre a equipe de desenvolvimento e o cliente, validando os resultados com os
	      \textit{stakeholders}.
	    
	    \item \textbf{Artefato(s) de saída}: Lista de termos técnicos, Entendimento do Negócio.
	      
	  \end{itemize}
     
     \item Atividade \textbf{Definir Tema de Investimento}
	
	\begin{itemize}
	  \item \textbf{Artefato(s) de entrada}: Entendimento do Negócio.
	  
	  \item \textbf{Descrição}: Esta atividade consiste em definir o Tema de Investimento da organização,
	    para a derivação dos épicos.
	  
	  \item \textbf{Artefato(s) de saída}: Tema de Investimento definido.
	 	 
	\end{itemize}
	
     \item Subprocesso \textbf{Identificar os épicos de negócio}
	
	\begin{itemize}
	  
	  \item Atividade \textbf{Elicitar os épicos}
	  
	      \begin{itemize}
		\item \textbf{Artefato(s) de entrada}: Tema de Investimento.
		
		\item \textbf{Descrição}: Esta atividade tem por objetivo abstrair do Tema de Investimento definido as
		  iniciativas de desenvolvimento em larga escala (épicos), juntamente com os \textit{stakeholders},
		  a partir das técnicas de elicitação definidas.
		
		\item \textbf{Artefato(s) de saída}: Épicos (iniciais).
		      
	      \end{itemize}
	      
	  \item Atividade \textbf{Validar os épicos}
	  
	      \begin{itemize}
		\item \textbf{Artefato(s) de entrada}: Épicos (iniciais).
		
		\item \textbf{Descrição}: Esta atividade tem por objetivo validar os épicos que foram elicitados, juntamente
		com os \textit{stakeholders}. O \textit{Backlog} do Portfólio será composto dos épicos validados pelos
		\textit{stakeholders}.
		
		\item \textbf{Artefato(s) de saída}: \textit{Backlog} do Portfólio.
		      
	      \end{itemize}
	\end{itemize} % Fim das atividades de um subprocesso
	      
      \item Atividade \textbf{Priorizar épico}
      
	  \begin{itemize}
	    \item \textbf{Artefato(s) de entrada}: \textit{Backlog}  do Portfólio.
	    
	    \item \textbf{Descrição}: Esta atividade tem por objetivo definir o épico de maior relevância de onde serão
	      abstraídas as \textit{features}.
	    
	    \item \textbf{Artefato(s) de saída}: Épico priorizado.
		  
	  \end{itemize}
	   
	
     \item Subprocesso \textbf{Gerenciar épicos}
     
	\begin{itemize}
	 
	 \item Atividade \textbf{Atualizar \textit{Backlog} do Portfólio}
	    
	    \begin{itemize}
	      \item \textbf{Artefato(s) de entrada}: \textit{Backlog} do Portfólio.

	      \item \textbf{Descrição}: Consiste no acompanhamento do \textit{Backlog} do Portfólio, atualizando eventuais
		mudanças e épicos que já foram implementados e revisados.
	      
	      \item \textbf{Artefato(s) de saída}: \textit{Backlog} do Portfólio (atualizado).
		    
	    \end{itemize}
	    
	 \item Atividade \textbf{Verificar mudanças nos épicos}
	    
	    \begin{itemize}
	      \item \textbf{Artefato(s) de entrada}: \textit{Backlog} do Portfólio.

	      \item \textbf{Descrição}: Consiste em verificar se são necessárias mudanças nos épicos do
		\textit{Backlog} do Portfólio.
	      
	      \item \textbf{Artefato(s) de saída}: \textit{Backlog} do Portfólio (atualizado), se houver mudanças.
		    
	    \end{itemize}
	    
	 \item Atividade \textbf{Analisar impactos}
	    
	    \begin{itemize}
	      \item \textbf{Artefato(s) de entrada}: Matriz de rastreabilidade.

	      \item \textbf{Descrição}: Consiste em analisar os impactos causados por uma mudança em um ou
		mais épicos, e mitigar o problema, se houver.
	      
	      \item \textbf{Artefato(s) de saída}: Matriz de rastreabilidade (atualizada).
		    
	    \end{itemize}
	    
	\end{itemize}
     
    \end{itemize}
    
    \vfill