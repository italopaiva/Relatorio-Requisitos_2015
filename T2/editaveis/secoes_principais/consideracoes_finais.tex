\chapter[Considerações finais]{Considerações finais}
    
  Nessa seção serão relatadas as experiências da equipe com a execução do trabalho, envolvendo a experiência com 
  a utilização das técnicas de elicitação, e com a disciplina de Engenharia de Requisitos em geral.
  
  \section{Experiência com a execução do trabalho}
  
    Na primeira parte da disciplina foi produzido um modelo de processo que precisávamos seguir na execução do trabalho dois. 
  O processo foi observado norteando nossas atividades e ações para que o resultado tivesse fundamento teórico observando 
  os compromissos firmados com a entrega do trabalho um.
    
    Definimos no primeiro trabalho duas técnicas de elicitação de requisitos: entrevista e workshop. 
  As duas técnicas foram aplicadas juntamente com o cliente para elitação de requisitos em vários 
  níveis e com várias finalidades, exemplificando a versatilidade das técnicas com relação ao contexto e com relação ao time.
    
    A ferramenta escolhida para gerenciamento dos requisitos foi a Gatherspace. Durante a execução da segunda 
  parte do trabalho a ferramenta foi utilizada para documentar aquilo que estava sendo executado, nela registramos 
  o tema de investimento, épicos, features e casos de uso do nosso trabalho com os atributos definidos no trabalho um.
    
    O cronograma é sempre uma parte importante em qualquer projeto, pois estima o tempo, esforço e atribui 
  responsabilidades aos integrantes do grupo. A nossa interação com o cronograma foi intensa, 
  pois a medida que o trabalho avançava o cronograma sempre precisava ser atualizado pois as estimativas de tempo e 
  esforço dos responsáveis nem sempre estavam de acordo com o que acontecia realmente. 
  Portanto o cronograma serviu de norte para a realização das tarefas ao passo que, mesmo quando os prazos não 
  estavam condizentes com a realidade, era necessário revisitar o cronograma e reavaliar as tarefas afim de 
  estimar com mais veracidade e proximidade com o real.
    
    Portanto, passando por todos os passos que nos comprometemos a passar no primeiro trabalho, 
  podemos dizer que alcançamos o objetivo, não com tanta facilidade, mas realizamos o processo e utilizamos as 
  técnicas e ferramentas pré-definidas para definir a solução.
  
  \section{Experiência com as técnicas de elicitação}
  
    As técnicas de elicitação foram um ponto de suma importância na preparação e execução do trabalho. 
  Elas, as técnicas, forneceram insumos para identificarmos características importantes do problema que nos 
  foi fornecido e as ferramentas necessárias para que pudéssemos lidar com essas características frente ao cliente.
    
    Na entrevista tivemos que nos organizar no que tange à agenda de reunião, visto que o grupo ainda não era 
  experiente com essa técnica. As agendas foram definidas e as entrevistas foram realizadas, algumas com mais 
  produtos e conteúdos gerados do que outras, a isso se deve ao fato do escopo da reunião não ser tão 
  bem definido quanto deveria, mas em um olhar geral sobre as entrevistas obtivemos um resultado satisfatório 
  levando à construção concreta e bem fundamentada da solução do problema.
    
    O workshop foi a técnica mais produtiva empregada nesta fase do trabalho. A característica de ser interativo 
  e dar voz a todos os participantes gerou resultados mais que satisfatórios e interações entre cliente e equipe 
  de desenvolvimento que estreitou os entendimentos das duas partes. A aplicação da técnica ocorreu de forma tranquila 
  e bem organizada, seus resultados geraram subsídio concreto para desenvolvimento do entendimento do problema e da solução. 
  
  \section{Experiência com a disciplina de Engenharia de Requisitos}
    
    A disciplina de Engenharia de Requisitos tem uma proposta que surpreende os estudades a início, 
  mas fornece os caminhos para que possamos chegar ao resultado esperado no final do processo e da disciplina.
    
    Nosso grupo foi formado ao acaso e nem todos integrantes haviam trabalhado com os outros, esse fato 
  dificultou na integração e trabalho em equipe, mas o grupo, mesmo com as diferenças, conseguiu agir em conjunto 
  para seguir o que havíamos nos proposto fazer.
    
    Na primeira parte da disciplina foi realizado um trabalho de pesquisa intensivo para produzir valor ao 
  trabalho à luz das referências escolhidas, gerando assim um processo consistente e adaptado ao nosso contexto e time. 
  A definição do processo foi um trabalho em equipe que mostrou ao grupo o potencial que temos e a capacidade de 
  transpor essas barreiras impostas pela falta de integração.
    
    A segunda parte da disciplina foi a execução do planejamento da primeira. Nosso contexto se mostrou de 
  complexidade considerável demandando várias reuniões com o cliente para total entendimento sobre o problema 
  possibilitando o planejamento da solução que o atendesse. O cliente confrontado com as técnicas de elicitação 
  foi bem elucidativo, sempre explicando os problemas e processo do contexto por meio de desenhos e esquemas 
  gráficos que facilitam o entendimento mútuo.
    
    Observando o desenvolver do nosso grupo na disciplina e o escopo da disciplina de Requisitos na Engenharia de software, 
  pudemos observar a importância dessa disciplina e do trabalho em equipe para alcançar um produto de software com qualidade 
  e uma equipe de desenvolvimento integrada e auto-gerenciável.
