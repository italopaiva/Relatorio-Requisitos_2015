\chapter{Execução à nível de Portfólio}
  
  Esse capítulo apresenta as atividades e resultados obtidos na execução do processo a nível de Portfólio.
  
  \section{Reuniões realizadas}
    
    Foram realizadas três reuniões entre a equipe de requisitos e o cliente, no contexto do nível de Portfólio. 
    
    \subsection{1ª reunião}
      
      Primeiramente, a equipe de requisitos analisou o contexto de negócio que foi proposto.
      Com a análise do contexto de negócio proposto, foi realizada a primeira reunião com o cliente, que tinha o objetivo de 
      obter um entendimento inicial do negócio por parte da equipe de requisitos. 
      
      Para obter o entendimento do negócio, a equipe de requisitos levantou perguntas com base no contexto de negócio apresentado, 
      de modo que as perguntas guiassem o rumo da entrevista.
      
      O objetivo da reunião foi atingido com o entendimento do negócio estabelecido entre as partes. Nessa reunião 
      foram levantados também alguns termos técnicos para o contexto da organização, que se encontram no documento de visão
      (em apêndice).
      
    \subsection{2ª reunião}
    
      Com o entendimento de negócio estabelecido, a equipe de requisitos analisou os processos e atividades da 
      organização para definir o problema da organização. A segunda reunião foi realizada para a validação do problema 
      da organização junto ao cliente.
      
      Nessa reunião ainda foi possível definir o Tema de Investimento da organização.
    
    \subsection{3ª reunião}
    
      Com o Tema de Investimento validado, a equipe de requisitos identificou possíveis épicos de negócio e foi realizada a terceira
      reunião utilizando a técnica de entrevista para o levantamento e validação dos épicos junto ao cliente.
      
      A equipe de requisitos criou perguntas para obter respostas do cliente que permitissem a avaliação dos épicos identificados.
      Nessa reunião também foi possível levantar algumas características do sistema.
    
  \section{Entendimento do negócio}
  
    \chapter{Entendimento do negócio}
    
  \section{Problema da organização}
    
    Através das reuniões da equipe de requisitos com o cliente, foram identificados aspectos inerentes ao funcionamento
atual do Corpo de Bombeiros Militar do Distrito Federal (CBMDF). Esses aspectos foram agrupados e deram origem a causas, estas
que ajudaram a identificar o problema do CBMDF.

As causas citadas foram representadas, junto ao problema, em um diagrama de causa e evento na figura \ref{fishbone}.

  \begin{figure}[!htbp]
    \centering
    \includegraphics[scale=0.8, angle=0]{figuras/fishbone}
    \caption{Diagrama de causa e evento (\textit{fishbone}) com as causas e problema do CBMDF.}
    \label{fishbone}
  \end{figure}

    \vfill
  \pagebreak

Para uma descrição mais clara e formal do problema, foi utilizado o seguinte \textit{framework} de descrição de problema:

  \textbf{O problema:} Ineficiência da gestão das informações acerca das viaturas.
  
  \textbf{Afeta:} Corpo de Bombeiros Militar do Distrito Federal.
  
  \textbf{Cujo impacto é:} Tomadas de decisões inadequadas, falta de qualidade de vida para os bombeiros e 
  gastos desnecessários.
  
  \textbf{Uma solução bem sucedida seria:} Um sistema que centralizaria as informações acerca das viaturas, as fornecendo 
  em tempo real para quem necessitasse das mesmas.
  
  \vfill
  \pagebreak
  
  \section{Considerações finais a nível de Portfólio}
    
    Esta seção apresenta um resumo dos itens mais importantes produzidos a nível de Portfólio.
    
    \subsection{Tema de investimento definido}
      
      O seguinte tema de investimento ficou definido para a organização:
      
      \emph{Tema de investimento (T1): Gestão de viaturas.}
    
    \subsection{Épicos identificados}
      
      O \textit{Backlog} do Portfólio ficou composto pelos seguintes épicos:
      
      \begin{itemize}
      \item \emph{Épico 1 (T1E1) – Gestão Operacional nas unidades;}
      \item \emph{Épico 2 (T1E2) – Gestão Administrativa nas unidades.}
      \end{itemize}

    