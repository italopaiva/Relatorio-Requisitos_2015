\chapter{Execução à nível de Time}

  Esse capítulo apresenta as atividades e resultados obtidos na execução do processo a nível de Time.
  
  \begin{figure}[!htbp]
    \centering
    \includegraphics[scale=0.33]{figuras/processo_time}
    \caption[Processo - Nível de Time]{Processo - Nível de Time.}
    \label{processo_time}
  \end{figure}
  
  \section{Reuniões realizadas}
  
    A nível de Time, foram realizadas duas reuniões com o cliente.
    
    Na útima reunião a nível de programa foi definido o escopo de implementação.
    Os casos de uso identificados que foram para o escopo de implementação foram priorizados 
    (seguindo a lógica utilizada para a construção do \textit{Roadmap}) e especificados juntos aos clientes 
    ao longo das reuniões que ocorreram no nível de Time.
    
    Inicialmente, os casos de uso responsáveis por manter os registros de motoristas, viaturas, abastecimentos, missões e escala
    não estavam decompostos em suas respectivas funcionalidades de cadastro, alteração, consulta e exclusão. A equipe optou por 
    decompor funcionalmente estes casos de uso porque algumas dessas funcionalidades básicas não podiam ser executadas por qualquer
    ator do sistema.
    
    Esta seção apresenta o resumo das reuniões que ocorreram no nível de Time.
    
    \subsection{Resumo da 1ª reunião}
           
      Nesta reunião foi utilizada a técnica de entrevista para a especificação dos casos de uso "Cadastrar motorista", 
      "Alterar motorista", "Consultar motorista", "Inativar motorista", "Cadastrar viatura", "Alterar viatura",
      "Consultar viatura", "Inativar viatura" e "Gerenciar vínculo entre motorista e viatura", junto ao cliente.
      
      A equipe de requisitos planejou a entrevista construindo um possível cenário de caso de uso para a posterior
      validação do cliente. Tomando este cenário como guia para a entrevista, todas as informações necessárias foram surgindo
      conforme o rumo da conversa ia mudando.
            
    \subsection{Resumo da 2ª reunião}
      
      Nesta reunião foram validados pontos não respondidos na reunião anterior e também foi feita a especificação dos casos de uso
      "Cadastrar missão", "Alterar missão", "Consultar missão" e "Gerenciar vínculo entre viaturas e missões".
      O planejamento da reunião seguiu o modelo da reunião anterior.
      
  
  \section{Especificação dos Casos de Uso}
 
    Foram especificados os casos de uso definidos para o escopo de implementação:
    
    \begin{itemize}
      \item Cadastrar missão;
      \item Alterar missão;
      \item Consultar missão;
      \item Gerenciar vínculo entre viaturas e missões
      \item Cadastrar motorista;
      \item Alterar motorista;
      \item Consultar motorista;
      \item Inativar motorista;
      \item Gerenciar vínculo entre motoristas e viaturas;
      \item Cadastrar viatura;
      \item Alterar viatura;
      \item Consultar viatura;
      \item Inativar viatura.
    \end{itemize}

    As especificações dos casos de usos se encontram no Apêndice \ref{Especificacao}.
  
  \section{Modelagem dos casos de uso}
    
    Essa seção apresenta a modelagem dos casos de uso realizados pela equipe de requisitos.
    
    \subsection{Diagrama de atores}
    
      Foi necessária uma generalização dos atores do sistema para melhor representar a interação destes com o sistema. As generalizações
      feitas está representada no diagrama de atores ilustrado na figura \ref{diagrama_de_atores}.
      
      \begin{figure}[!htbp]
	\centering
	\includegraphics[scale=1]{figuras/diagrama_de_atores}
	\caption[Diagrama de atores]{Diagrama de atores.}
	\label{diagrama_de_atores}
      \end{figure}
    
    \vfill
    \pagebreak
    \subsection{Diagrama de casos de uso}
      
      A figura \ref{diagrama_de_casos_uso} ilustra o diagrama de casos de uso modelado para o sistema.
      
      \begin{figure}[!htbp]
	\centering
	\includegraphics[scale=0.85]{figuras/diagrama_de_casos_uso}
	\caption[Diagrama de casos de uso para o sistema]{Diagrama de casos de uso para o sistema.}
	\label{diagrama_de_casos_uso}
      \end{figure}
    
    \subsection{Diagrama de casos de uso para o sistema \textit{mobile}}
      
      A figura \ref{caso_de_uso_mobile} ilustra, a parte, o diagrama de casos de uso modelado para o sistema \textit{mobile}.
      
      \begin{figure}[!htbp]
	\centering
	\includegraphics[scale=1]{figuras/caso_de_uso_mobile}
	\caption[Diagrama de casos de uso para o sistema \textit{mobile}]{Diagrama de casos de uso para o sistema \textit{mobile}.}
	\label{caso_de_uso_mobile}
      \end{figure}
    
  \section{\textit{Sprints}}
    
    Esta seção apresenta os resultados da execução das \textit{sprints}.
    
    \subsection{Planejamento da \textit{Sprint 1}}
      
	Essa seção apresenta o planejamento feito para a \textit{sprint} 1.
      
      \subsubsection{\textit{Backlog} da \textit{Sprint}}
	
	Os seguintes casos de comporão o \textit{Backlog da sprint}:
	
	\begin{itemize}
	 \item Cadastrar motorista;
	 \item Alterar motorista;
	 \item Consultar motorista;
	 \item Cadastrar viatura;
	 \item Alterar viatura;
	 \item Consultar viatura;
	 \item Gerenciar vínculo entre motoristas e viaturas;
	\end{itemize}
  
  \section{Considerações finais a nível de Time}
    
    Esta seção apresenta um resumo dos itens mais importantes produzidos a nível de Time.
    
    \subsection{Casos de uso identificados} 
      
      O \textit{Backlog} do Time ficou composto pelos seguintes casos de uso:
      
      \begin{itemize}

    \item \textbf{F1UC1 - Cadastrar missão}
      \subitem
      Este caso de uso permite ao Sargento de uma unidade do CBM-DF cadastrar dados de uma missão no sistema de gestão de viaturas para
      manter os registros no sistema.

    \item \textbf{F1UC2 - Alterar missão}
      \subitem
      Este caso de uso permite ao Sargento de uma unidade do CBM-DF alterar dados de uma missão no sistema de gestão de viaturas
      para manter o registro no sistema.

    \item \textbf{F1UC3 - Consultar missão}
      \subitem
      Este caso de uso permite ao Consultor operacional do CBM-DF consultar dados de missões no sistema de gestão de viaturas para
      manter o registro no sistema.

    \item \textbf{F1UC4 - Gerenciar vínculo entre viaturas e missão}
      \subitem
      Este caso de uso permite ao Sargento de uma unidade do CBM-DF associar uma viatura a missão no sistema de gestão de viaturas 
      para manter o registro no sistema.

    \item \textbf{F1UC5 - Gerar relatório da missão}
      \subitem
      Este caso de uso permite a um funcionário da gestão administrativa das unidades ou do Gestor Operacional gerar relatórios de
      missões realizadas para verificar os dados existentes.

    \item \textbf{F2UC1 - Adicionar abastecimento}
      \subitem
      Este caso de uso permite a adição de dados de abastecimento das viaturas das unidades do CBM-DF para manter o registro no sistema.

    \item \textbf{F2UC2 - Alterar abastecimento}
      \subitem
      Este caso de uso permite a alteração de dados de abastecimento das viaturas das unidades do CBM-DF para manter o registro no sistema.

    \item \textbf{F2UC3 - Consultar abastecimento}
      \subitem
      Este caso de uso permite a consulta dos dados de abastecimento das viaturas das unidades do CBM-DF para verificar os registros
      do sistema.

    \item \textbf{F2UC4 - Inserir recibo do abastecimento}
      \subitem
      Este caso de uso permite a inserção do recibo de abastecimento nos dados do abastecimento, quando este for cadastrado no 
      sistema de gerenciamento de viaturas do CBM-DF para manter o registro no sistema.

    \item \textbf{F2UC5 - Gerar relatório do consumo de combustível}
      \subitem
      Este caso de uso permite a um funcionário da gestão administrativa das unidades ou o Gestor Operacional gerar relatórios de
      consumo de combustível por tipo de combustível, postos, motoristas, tipos de viaturas para verificar os dados existentes. 

    \item \textbf{F3UC1 - Consultar viaturas disponíveis nas unidades}
      \subitem
      Este caso de uso permite a um Gestor Operacional consultar viaturas disponíveis em uma unidade do CBM-DF para verificar os 
      dados existentes no sistema.
  
    \item \textbf{F3UC2 - Visualizar unidades no mapa}
      \subitem
      Este caso de uso permite a um Gestor Operacional consultar unidades do CBM-DF no mapa para verificar os dados existentes no sistema.

    \item \textbf{F4UC1 - Criar escala}
      \subitem
      Este caso de uso permite ao funcionário da gestão administrativa das unidades adicionar uma escala no sistema de
      gerenciamento de viaturas do CBM-DF para manter o registro no sistema.

    \item \textbf{F4UC2 - Alterar escala}
      \subitem
      Este caso de uso permite ao funcionário da gestão administrativa das unidades alterar uma escala no sistema de
      gerenciamento de viaturas do CBM-DF para manter o registro no sistema.

    \item \textbf{F4UC3 - Consultar escala}
      \subitem
      Este caso de uso permite ao funcionário da gestão administrativa das unidades consultar uma escala no sistema 
      de gerenciamento de viaturas do CBM-DF para visualizar os registros do sistema.

    \item \textbf{F4UC4 - Excluir escala}
      \subitem
      Este caso de uso permite ao funcionário da gestão administrativa das unidades  excluir uma escala no sistema 
      de gerenciamento de viaturas do CBM-DF para deletar o registro do sistema.

    \item \textbf{F4UC5 - Cadastrar motorista}
      \subitem
      Este caso de uso permite que o Gestor Operacional cadastre um motorista na corporação do CBM-DF para manter o registro no sistema.

    \item \textbf{F4UC6 - Consultar motorista}
      \subitem
      Este caso de uso permite que o Gestor Operacional consulte um motorista na corporação do CBM-DF para verificar o registro no sistema.

    \item \textbf{F4UC7 - Alterar motorista}
      \subitem
      Este caso de uso permite que o Gestor Operacional altere um motorista na corporação do CBM-DF para manter o registro no sistema.

    \item \textbf{F4UC8 - Inativar motorista}
      \subitem
      Este caso de uso permite que o Gestor Operacional inative um motorista na corporação do CBM-DF para desativar o registro no sistema.

    \item \textbf{F4UC9 - Gerenciar vínculo entre motoristas e viaturas}
      \subitem
      Este caso de uso permite a um Sargento vincular um motorista à uma viatura ou desvincular um motorista de uma viatura da unidade
      para manter o registro no sistema.

    \item \textbf{F4UC10 - Gerar relatórios sobre dados dos motoristas}
      \subitem
      Este caso de uso permite a um funcionário da gestão administrativa das unidades ou d]o Gestor Operacional gerar relatórios de 
      motoristas por tipo de habilitação, tempo de vencimento de habilitação, nome, cursos completados e status para visualizar
      o registro do sistema.

    \item \textbf{F4UC11 - Alocar motorista à unidade}
      \subitem
      Este caso de uso permite que o Gestor Operacional designe um motorista a uma unidade do CBM-DF para manter o registro no sistema.

    \item \textbf{F5UC1 - Cadastrar viaturas}
      \subitem
      Este caso de uso permite que o Gestor Operacional adicione viaturas ao sistema de gerenciamento de viaturas do CBM-DF 
      para manter o registro no sistema.

    \item \textbf{F5UC2 - Consultar viaturas}
      \subitem
      Este caso de uso permite a consulta de viaturas no sistema de gerenciamento de viaturas do CBM-DF para manter o registro no sistema.

    \item \textbf{F5UC3 - Alterar dados das viaturas}
      \subitem
      Este caso de uso permite que o Gestor Operacional altere dados de viaturas no sistema de gerenciamento de viaturas do 
      CBM-DF para manter o registro no sistema.

    \item \textbf{F5UC4 - Inativar viaturas}
      \subitem
      Este caso de uso permite que o Gestor Operacional desative viaturas no sistema de gerenciamento de viaturas do CBM-DF para 
      manter o registro no sistema.

    \item \textbf{F5UC5 - Alocar viatura em uma unidade}
      \subitem
      Este caso de uso permite que o Gestor Operacional atribua uma unidade de operação específica a uma viatura para manter o 
      registro no sistema.

    \item \textbf{F5UC6 - Estabelecer custo do combustível}
      \subitem
      Este caso de uso permite que o Gestor Operacional estabeleça e modifique o custo do combustível para abastecimento
      das viaturas do CBM-DF para manter o registro no sistema.

    \item \textbf{F5UC7 - Gerar relatórios referentes a dados das viaturas}
      \subitem
      Este caso de uso permite a um funcionário da gestão administrativa das unidades gerar relatórios de viaturas das
      unidades ou o Gestor Operacional gerar relatórios de todas as viaturas de todas as unidades por tipo de viaturas,
      estado da viatura e unidades de alocação para visualizar o registro do sistema.

\end{itemize}
