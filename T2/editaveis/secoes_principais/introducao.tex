\chapter[Introdução]{Introdução}
  
%   Será apresentado neste capítulo uma breve contextualização do projeto contendo o contexto do negócio, planejamento do projeto e
%   processo de Engenharia de Requisitos e abordagem utilizados.
  
  Neste trabalho será apresentado o resultado da execução do processo de Engenharia de Requisitos elaborado no
  planejamento feito no trabalho 1. O cronograma da fase de execução do processo se encontra no apêndice \ref{cronograma_execucao}.
  
  O contexto de negócio que norteia o projeto é o problema que o Corpo de Bombeiros do Distrito Federal (CBMDF)
  tem para gerenciar sua frota de viaturas, onde grande parte do trabalho é feito manualmente.
  
  A partir da análise do contexto de negócio e do contexto da organização, foi definida uma abordagem híbrida do método
  ágil proposto no \textit{Scaled Agile Framework} (SAFe) junto com o método tradicional do \textit{Rational Unified Process} (RUP).
  
  Com a abordagem definida, foi desenhado o processo de Engenharia de Requisitos com as atividades necessárias para a criação da
  solução de \textit{software} para o problema da organização. A Figura ~\ref{project_big_picture} (em apêndice) ilustra o modelo do
  processo desenvolvido.
  
  Para a elicitação dos requisitos com o cliente foram escolhidas as técnicas de entrevista e \textit{workshop}.
  
  A ferramenta definida para gerenciar os requisitos foi a \textit{GatherSpace}.
  
  \vfill
  
  