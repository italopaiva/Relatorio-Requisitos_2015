
  Os papéis, e responsabilidades dos papéis, definidos para o processo são os seguintes:
  
  \begin{itemize}
   
   \item \textbf{\textit{Especialista do Negócio}}
   
      O Especialista do Negócio é o \textit{stakeholder} que detém o conhecimento do negócio, do contexto organizacional
      e da visão do produto. O Especialista de Negócio será o responsável por fazer todas as validações necessárias, uma vez que
      ele é quem detém o conhecimento necessário para tal.
      
      O professor atuará como Especialista do Negócio, majoritariamente, em todos os níveis (Portfólio, Programa e Time).
   
   \item \textbf{\textit{Product Owner}}
      
      Segundo conceitos apresentados por Leffingwell (\citeyear{leffingwell11}),
      são atividades de responsabilidade do \textit{Product Owner} (PO) e do \textit{Product Manager} (PM), 
      adaptados para o processo do projeto, respectivamente:
      
      \textbf{PO:}
      
      \begin{itemize}
       
       \item Trabalhar com o PM;
       
       \item Definir objetivos para a \textit{sprint};
       
       \item Priorizar e manter o \textit{Backlog} do Time;
       
       \item Elaborar e validar os casos de uso;
       
       \item Participar do planejamento da \textit{sprint} e validar a \textit{sprint}.
       
      \end{itemize}
      
      \textbf{PM:}
      
      \begin{itemize}
       
       \item Manter a Visão e o \textit{Backlog} do Programa;
       
       \item Priorizar \textit{features} e manter o \textit{Roadmap};
       
       \item Gerenciar o conteúdo da \textit{release};
       
       \item Manter e priorizar o \textit{Backlog} do Portfólio;
       
      \end{itemize}
      
      Como as responsabilidades de ambos papéis serão da equipe de requisitos, exceto as atividades de validações que serão
      realizadas pelo Especialista de Negócio, ficou definido o papel do PO para tratar dessas responsabilidades nos três
      níveis (Portfólio, Programa e Time).
      
   \item \textbf{\textit{Scrum Master}}
      
      O \textit{Scrum Master} é responsável por assistir o time para garantir a melhor perfomance, atuando com 
      um líder do time \cite{leffingwell11}.
      São responsabilidades do \textit{Scrum Master}, segundo Leffingwell (\citeyear{leffingwell11}):
      
      \begin{itemize}
       
       \item Facilitar o progresso do time;
       
       \item Liderar os esforços do time;
       
       \item Eliminar impedimentos;
       
      \end{itemize}
      
      O papel do \textit{Scrum Master} será realizado pelo integrante da equipe de requisitos, Matheus Silva.
      
   \item \textbf{Time}
      
      O time é composto pelos desenvolvedores, que é representado por toda a equipe de requisitos. Este papel é realizado
      a nível de Time.
      
  \end{itemize}
  
  \vfill