  
  \subsection{Artefatos envolvidos}
  
    São listados abaixo os conceitos de alguns artefatos utilizados no processo, no contexto do projeto,
    para o melhor entendimento das atividades do processo.
    
    \begin{itemize}
     
     \item \textbf{Temas de Investimento} - Representa o conjunto de iniciativas que coordenam o portfólio da organização \cite{leffingwell11}.
      Os épicos são derivados a partir de um Tema de Investimento.
     
     \item \textbf{Épicos} - São iniciativas de desenvolvimento em larga escala que agregam valor a um tema de investimento \cite{leffingwell11}.
     São os artefatos de requisitos de mais alto nível no processo \cite{leffingwell11}.
     São decompostos em \textit{features}.
     
     \item \textbf{\textit{Backlog} do Portfólio} - É o repositório dos épicos.
     
     \item \textbf{\textit{Features}} - São serviços fornecidos pelo sistema que atendem às necessidades dos
     \textit{stakeholders} \cite{leffingwell03}. As \textit{features} atuam como pontes entre as necessidades
     dos \textit{stakeholders} (requisitos de alto nível) e os requisitos específicos no domínio da solução \cite{leffingwell11}.
     São decompostas em casos de uso.
     
     \item \textbf{Requisitos não-funcionais} - São qualidades do sistema, que caracterizam o comportamento do mesmo.
      Serão armazenados no Documento de Visão do projeto e serão escritos baseado no modelo de Especificações Suplementares
      fornecido por Leffingwell e Widrig (\citeyear{leffingwell03}).
     
     \item \textbf{\textit{Backlog} do Programa} - É o repositório das \textit{features} do sistema.
     
     \item \textbf{\textit{Roadmap}} - Consiste em um conjunto de \textit{releases} com datas planejadas, cada uma com um tema,
     um conjunto de \textit{features} priorizadas e um conjunto de objetivos, demonstrando a intenção da empresa em mostrar valor ao
     longo do tempo \cite{leffingwell11}.
     
     \item \textbf{Documento de Visão} - É um mecanismo para definir e comunicar a Visão do sistema \cite{leffingwell11}.
      O conteúdo primário da Visão de um sistema é o conjunto de \textit{features} priorizadas que descrevem o que o sistema
      será capaz de oferecer a seus usuários, para atender as necessidades dos envolvidos \cite{leffingwell11}.
      O Documento de Visão do projeto armazenará também os requisitos não-funcionais e os termos técnicos identificados na
      análise do negócio, entre outras informações importantes.
 
     \item \textbf{Casos de Uso} - Um caso de uso captura um contrato sobre o comportamento de um sistema
      entre os \textit{stakeholders} desse sistema \cite{cockburn01}.
      Descreve o comportamento do sistema frente a várias condições, enquanto o sistema responde a
      um pedido de um dos \textit{stakeholders} \cite{cockburn01}. Os casos de uso fornecem parte do insumo para a implementação.
      
     \item \textbf{\textit{Backlog} do Time} - É o repositório dos casos de uso que serão desenvolvidos em uma \textit{release}.
     
     \item \textbf{\textit{Backlog} da Sprint} - É o repositório dos casos de uso que serão desenvolvidos em uma \textit{sprint}.
     
     \item \textbf{Incremento de \textit{software}} - É a porção do \textit{software} que é gerado em cada \textit{sprint}.
     
     \item \textbf{\textit{Build}} - É o conjunto dos incrementos de \textit{software} que foram gerados em cada \textit{sprint}
      durante uma \textit{release}.
     
    \end{itemize}

  \subsection{Atividades e seus artefatos}
    
    A seguir são descritas as atividades e os artefatos envolvidos no processo em cada nível do SAFe, Portfólio, Programa e Time.
    
    
  \subsubsection{Nível de Portfólio}
    
    \begin{itemize}
      
      \item Atividade \textbf{Analisar o negócio}
	  
	  \begin{itemize}
	    \item \textbf{Artefato(s) de entrada}: Contexto de negócio, Processo atual de negócio.
	    
	    \item \textbf{Descrição}: Esta atividade tem por objetivo entender o contexto do negócio, os problemas de mais alto nível
	      e estabelecer um vocabulário comum entre a equipe de desenvolvimento e o cliente, validando os resultados com os
	      \textit{stakeholders}.
	    
	    \item \textbf{Artefato(s) de saída}: Lista de termos técnicos, Entendimento do Negócio.
	      
	  \end{itemize}
     
     \item Atividade \textbf{Definir Tema de Investimento}
	
	\begin{itemize}
	  \item \textbf{Artefato(s) de entrada}: Entendimento do Negócio.
	  
	  \item \textbf{Descrição}: Esta atividade consiste em definir o Tema de Investimento da organização,
	    para a derivação dos épicos.
	  
	  \item \textbf{Artefato(s) de saída}: Tema de Investimento definido.
	 	 
	\end{itemize}
	
     \item Subprocesso \textbf{Identificar os épicos de negócio}
	
	\begin{itemize}
	  
	  \item Atividade \textbf{Elicitar os épicos}
	  
	      \begin{itemize}
		\item \textbf{Artefato(s) de entrada}: Tema de Investimento.
		
		\item \textbf{Descrição}: Esta atividade tem por objetivo abstrair do Tema de Investimento definido as
		  iniciativas de desenvolvimento em larga escala (épicos), juntamente com os \textit{stakeholders},
		  a partir das técnicas de elicitação definidas.
		
		\item \textbf{Artefato(s) de saída}: Épicos (iniciais).
		      
	      \end{itemize}
	      
	  \item Atividade \textbf{Validar os épicos}
	  
	      \begin{itemize}
		\item \textbf{Artefato(s) de entrada}: Épicos (iniciais).
		
		\item \textbf{Descrição}: Esta atividade tem por objetivo validar os épicos que foram elicitados, juntamente
		com os \textit{stakeholders}. O \textit{Backlog} do Portfólio será composto dos épicos validados pelos
		\textit{stakeholders}.
		
		\item \textbf{Artefato(s) de saída}: \textit{Backlog} do Portfólio.
		      
	      \end{itemize}
	\end{itemize} % Fim das atividades de um subprocesso
	      
      \item Atividade \textbf{Priorizar épico}
      
	  \begin{itemize}
	    \item \textbf{Artefato(s) de entrada}: \textit{Backlog}  do Portfólio.
	    
	    \item \textbf{Descrição}: Esta atividade tem por objetivo definir o épico de maior relevância de onde serão
	      abstraídas as \textit{features}.
	    
	    \item \textbf{Artefato(s) de saída}: Épico priorizado.
		  
	  \end{itemize}
	   
	
     \item Subprocesso \textbf{Gerenciar épicos}
     
	\begin{itemize}
	 
	 \item Atividade \textbf{Atualizar \textit{Backlog} do Portfólio}
	    
	    \begin{itemize}
	      \item \textbf{Artefato(s) de entrada}: \textit{Backlog} do Portfólio.

	      \item \textbf{Descrição}: Consiste no acompanhamento do \textit{Backlog} do Portfólio, atualizando eventuais
		mudanças e épicos que já foram implementados e revisados.
	      
	      \item \textbf{Artefato(s) de saída}: \textit{Backlog} do Portfólio (atualizado).
		    
	    \end{itemize}
	    
	 \item Atividade \textbf{Verificar mudanças nos épicos}
	    
	    \begin{itemize}
	      \item \textbf{Artefato(s) de entrada}: \textit{Backlog} do Portfólio.

	      \item \textbf{Descrição}: Consiste em verificar se são necessárias mudanças nos épicos do
		\textit{Backlog} do Portfólio.
	      
	      \item \textbf{Artefato(s) de saída}: \textit{Backlog} do Portfólio (atualizado), se houver mudanças.
		    
	    \end{itemize}
	    
	 \item Atividade \textbf{Analisar impactos}
	    
	    \begin{itemize}
	      \item \textbf{Artefato(s) de entrada}: Matriz de rastreabilidade.

	      \item \textbf{Descrição}: Consiste em analisar os impactos causados por uma mudança em um ou
		mais épicos, e mitigar o problema, se houver.
	      
	      \item \textbf{Artefato(s) de saída}: Matriz de rastreabilidade (atualizada).
		    
	    \end{itemize}
	    
	\end{itemize}
     
    \end{itemize}
    
    \vfill
    
    \pagebreak
    
  \subsubsection{Nível de Programa}
    
    \begin{itemize}
	
     \item Subprocesso \textbf{Levantar as \textit{features}}
	
	\begin{itemize}
	  \item Atividade \textbf{Elicitar as \textit{features}}
	      
	      \begin{itemize}
		  \item \textbf{Artefato(s) de entrada}: Épico priorizado.
		  
		  \item \textbf{Descrição}: Esta atividade tem por objetivo abstrair do épico priorizado
		    as \textit{features} necessárias, juntamente com os \textit{stakeholders}, a partir das
		    técnicas de elicitação definidas.
		  
		  \item \textbf{Artefato(s) de saída}: \textit{Features} (iniciais).
		    
		\end{itemize}
	    
	    \item Atividade \textbf{Validar as \textit{features}}
	    
		\begin{itemize}
		  \item \textbf{Artefato(s) de entrada}: \textit{Features} (iniciais).
		  
		  \item \textbf{Descrição}: Esta atividade consiste em validar as \textit{features} elicitadas,
		    juntamente com os \textit{stakeholders}. O \textit{Backlog} do Programa será composto
		    pelas \textit{features} validadas pelos \textit{stakeholders}.
		  
		  \item \textbf{Artefato(s) de saída}: \textit{Backlog} do Programa.
			
		\end{itemize}
	\end{itemize} % Fim das atividades de um subprocesso
	      
     \item Atividade \textbf{Identificar os Requisitos não-funcionais}
      
	  \begin{itemize}
	    \item \textbf{Artefato(s) de entrada}: Nenhum.
	    
	    \item \textbf{Descrição}: Esta atividade tem por objetivo abstrair das necessidades do cliente os
	      requisitos não-funcionais do sistema. Se já houver requisitos não-funcionais identificados para o sistema,
	      os mesmos deverão ser consultados para atualizações.
	    
	    \item \textbf{Artefato(s) de saída}: Requisitos não-funcionais.
		  
	  \end{itemize}
	  
     \item Atividade \textbf{Elaborar \textit{Roadmap}}
      
	  \begin{itemize}
	    \item \textbf{Artefato(s) de entrada}: \textit{Backlog} do Programa.
	    
	    \item \textbf{Descrição}: Esta atividade tem por objetivo priorizar as \textit{features} do
	      \textit{Backlog} do Programa, planejando o arranjo das mesmas nas \textit{releases}.  Caso já exista
	      um \textit{Roadmap}, o mesmo deve ser atualizado com eventuais mudanças.
	    
	    \item \textbf{Artefato(s) de saída}: \textit{Roadmap}.
		  
	  \end{itemize}
	  
     \item Atividade \textbf{Definir Visão}
      
	  \begin{itemize}
	    \item \textbf{Artefato(s) de entrada}: Lista de termos técnicos, Requisitos não-funcionais e \textit{Roadmap}.
	    
	    \item \textbf{Descrição}: Esta atividade tem por objetivo elaborar (ou atualizar, caso já exista um
	      documento de visão) o documento de visão.
	    
	    \item \textbf{Artefato(s) de saída}: Documento de Visão.
		  
	  \end{itemize}
	  
     \item Atividade \textbf{Priorizar uma \textit{feature}}
      
	  \begin{itemize}
	    \item \textbf{Artefato(s) de entrada}: \textit{Roadmap}.
	    
	    \item \textbf{Descrição}: Esta atividade consiste em definir a \textit{feature} de maior relevância
	      de onde serão derivados os casos de uso.
	    
	    \item \textbf{Artefato(s) de saída}: \textit{Feature} priorizada.
		  
	  \end{itemize}
	
     \item Subprocesso \textbf{Gerenciar \textit{features}}
     
	\begin{itemize}
	 
	 \item Atividade \textbf{Atualizar \textit{Backlog} do Programa}
	    
	    \begin{itemize}
	      \item \textbf{Artefato(s) de entrada}: \textit{Backlog} do Programa.

	      \item \textbf{Descrição}: Consiste no acompanhamento do \textit{Backlog} do Programa, atualizando
		eventuais mudanças e \textit{features} que já foram implementadas e revisadas.
	      
	      \item \textbf{Artefato(s) de saída}: \textit{Backlog} do Programa (atualizado).
		    
	    \end{itemize}
	    
	 \item Atividade \textbf{Verificar mudanças nas \textit{features}}
	    
	    \begin{itemize}
	      \item \textbf{Artefato(s) de entrada}: \textit{Backlog} do Programa.

	      \item \textbf{Descrição}: Consiste em verificar se são necessárias mudanças nas \textit{features} do
		\textit{Backlog} do Programa.
	      
	      \item \textbf{Artefato(s) de saída}: \textit{Backlog} do Programa (atualizado), se houver mudanças.
		    
	    \end{itemize}
	    
	 \item Atividade \textbf{Analisar impactos}
	    
	    \begin{itemize}
	      \item \textbf{Artefato(s) de entrada}: Matriz de rastreabilidade.

	      \item \textbf{Descrição}: Consiste em analisar os impactos causados por uma mudança em uma ou mais
		\textit{features}, e mitigar o problema, se houver.
	      
	      \item \textbf{Artefato(s) de saída}: Matriz de rastreabilidade (atualizada).
		    
	    \end{itemize}
	    
	\end{itemize}
	
     \item Atividade \textbf{Restrospectiva da \textit{release}}
      
	  \begin{itemize}
	    \item \textbf{Artefato(s) de entrada}: \textit{Backlog} do Time e
	      Incremento de \textit{software} (\textit{Build} produzida pelo time na \textit{release}).
	    
	    \item \textbf{Descrição}: Esta atividade tem por objetivo validar o que foi produzido na \textit{release},
	      juntamente com os envolvidos.
	    
	    \item \textbf{Artefato(s) de saída}: Resumo da retrospectiva da \textit{release}.
		  
	  \end{itemize}
     
    \end{itemize}
    
    \pagebreak
    
  \subsubsection{Nível de Time}
    
    \begin{itemize}
	
     \item Subprocesso \textbf{Especificar Casos de Uso}
	
	\begin{itemize}
	  \item Atividade \textbf{Elicitar casos de uso}
	      
	      \begin{itemize}
		  \item \textbf{Artefato(s) de entrada}: \textit{Feature} priorizada.
		  
		  \item \textbf{Descrição}: Consiste em levantar os casos de uso juntamente com os clientes
		    (\textit{stakeholders}), utilizando as técnicas de elicitação definidas,  a partir de determinada
		    \textit{feature}.
		  
		  \item \textbf{Artefato(s) de saída}: Casos de uso (iniciais).
		    
		\end{itemize}
	    
	    \item Atividade \textbf{Validar casos de uso}.
	    
		\begin{itemize}
		  \item \textbf{Artefato(s) de entrada}: Casos de uso (iniciais).
		  
		  \item \textbf{Descrição}: Consiste em validar os casos de uso elicitados, juntamente com os
		    clientes (\textit{stakeholders}).
		  
		  \item \textbf{Artefato(s) de saída}: \textit{Backlog} do Time com os casos de uso que foram validados.
			
		\end{itemize}
		
	    \item Atividade \textbf{Escrever a descrição dos casos de uso}.
	    
		\begin{itemize}
		  \item \textbf{Artefato(s) de entrada}: \textit{Backlog} do Time.
		  
		  \item \textbf{Descrição}: Consiste em descrever detalhadamente os casos de uso do \textit{Backlog} do time.
		  
		  \item \textbf{Artefato(s) de saída}: Descrição dos casos de uso.
			
		\end{itemize}
		
	    \item Atividade \textbf{Elaborar diagrama de casos de uso}.
	    
		\begin{itemize}
		  \item \textbf{Artefato(s) de entrada}: Casos de uso (iniciais).
		  
		  \item \textbf{Descrição}: Consiste em modelar (UML) os casos de uso do \textit{Backlog} do time.
		  
		  \item \textbf{Artefato(s) de saída}: Diagrama de casos de uso.
			
		\end{itemize}
		
	    \item Atividade \textbf{Elaborar modelos de casos de uso}.
	    
		\begin{itemize}
		  \item \textbf{Artefato(s) de entrada}: Descrição de casos de uso e Diagrama de casos de uso.
		  
		  \item \textbf{Descrição}: Consiste em construir o modelo de casos de uso, juntando a descrição dos
		    casos de uso e o diagrama de casos de uso.
		  
		  \item \textbf{Artefato(s) de saída}: Modelos de casos de uso.
			
		\end{itemize}
	\end{itemize} % Fim das atividades de um subprocesso
	      
     \item Atividade \textbf{Priorizar casos de uso}.
      
	  \begin{itemize}
	    \item \textbf{Artefato(s) de entrada}: \textit{Backlog} do Time.
	    
	    \item \textbf{Descrição}: Consiste em priorizar os casos de uso do \textit{Backlog} do Time para
	      encaixá-los em uma \textit{release} (conjunto de \textit{sprints}).
	    
	    \item \textbf{Artefato(s) de saída}: \textit{Backlog} do Time (priorizado).
		  
	  \end{itemize}
	  
     \item Atividade \textbf{Planejar \textit{sprint}}
      
	  \begin{itemize}
	    \item \textbf{Artefato(s) de entrada}: \textit{Backlog} do Programa.
	    
	    \item \textbf{Descrição}: Consiste em selecionar os casos de uso do \textit{Backlog} do time que serão
	      implementados na \textit{sprint}, criando o \textit{Backlog} da \textit{sprint}, e dividir as
	      responsabilidades de desenvolvimento para os integrantes do time.
	    
	    \item \textbf{Artefato(s) de saída}: \textit{Backlog} da \textit{sprint}.
		  
	  \end{itemize}
	  
     \item Atividade \textbf{Executar \textit{sprint}}
      
	  \begin{itemize}
	    \item \textbf{Artefato(s) de entrada}: \textit{Backlog} da \textit{sprint}.
	    
	    \item \textbf{Descrição}: Consiste em implementar os casos de uso do \textit{Backlog} da \textit{sprint}
	      e validar o que foi feito a cada término com o time, além das reuniões diárias (\textit{stand-up meeting}).
	    
	    \item \textbf{Artefato(s) de saída}: Incremento de \textit{software} (\textit{Build}).
		  
	  \end{itemize}
	  
     \item Atividade \textbf{Fazer restrospectiva da \textit{sprint}}
      
	  \begin{itemize}
	    \item \textbf{Artefato(s) de entrada}: \textit{Backlog} da \textit{sprint}.
	    
	    \item \textbf{Descrição}: Consiste numa reunião do time para avaliar o que foi feito e analisar pontos
	      positivos, pontos negativos e possíveis melhorias para a próxima \textit{sprint}.
	    
	    \item \textbf{Artefato(s) de saída}: Resumo da retrospectiva, com os pontos levantados pela equipe
	     (possível \textit{KanBan} de retrospectiva).
		  
	  \end{itemize}
	  
     \item Atividade \textbf{Fazer revisão da \textit{sprint}}
      
	  \begin{itemize}
	    \item \textbf{Artefato(s) de entrada}: \textit{Backlog} da \textit{sprint}, \textit{Backlog} do time,
	      Incremento de \textit{software} gerado na \textit{sprint} (\textit{Build}).
	    
	    \item \textbf{Descrição}: Consiste numa reunião com os envolvidos para avaliar o que foi gerado na \textit{sprint}.
	      Permite alinhar a visão do time com a dos \textit{stakeholders}, verificando se o que foi feito está
	      alinhado com os propósitos iniciais e se haverá mudanças no que será feita em seguida (\textit{Backlog} do time).
	      	    
	    \item \textbf{Artefato(s) de saída}: Resumo da revisão da \textit{sprint}.
		  
	  \end{itemize}
	  
      \item Atividade \textbf{Gerenciar a \textit{sprint}}
      
	  \begin{itemize}
	    \item \textbf{Artefato(s) de entrada}: \textit{Backlog} da \textit{sprint}.
	    
	    \item \textbf{Descrição}: Consiste no acompanhamento e suporte do time durante a \textit{sprint},
	      gerindo o desenvolvimento e possíveis mudanças.
	    
	    \item \textbf{Artefato(s) de saída}: Nenhum.
		  
	  \end{itemize}
	
     \item Subprocesso \textbf{Gerenciar casos de uso}
     
	\begin{itemize}
	 
	 \item Atividade \textbf{Atualizar \textit{Backlog} do Time}
	    
	    \begin{itemize}
	      \item \textbf{Artefato(s) de entrada}: \textit{Backlog} do Time.

	      \item \textbf{Descrição}: Consiste no acompanhamento do \textit{Backlog} do Time, atualizando eventuais
		mudanças e casos de uso que já foram implementados e revisados.
	      
	      \item \textbf{Artefato(s) de saída}: \textit{Backlog} do Time (atualizado).
		    
	    \end{itemize}
	    
	 \item Atividade \textbf{Verificar mudanças nos casos de uso}
	    
	    \begin{itemize}
	      \item \textbf{Artefato(s) de entrada}: \textit{Backlog} do Time e Resumo da revisão de \textit{sprint}.

	      \item \textbf{Descrição}: Consiste em verificar se são necessárias mudanças nos casos de uso do \textit{Backlog}
		do Time. Pode acontecer mudanças dependendo do resultado da revisão de \textit{sprint}.
	      
	      \item \textbf{Artefato(s) de saída}: \textit{Backlog} do Time (atualizado).
		    
	    \end{itemize}
	    
	 \item Atividade \textbf{Analisar impactos}
	    
	    \begin{itemize}
	      \item \textbf{Artefato(s) de entrada}: Matriz de rastreabilidade.

	      \item \textbf{Descrição}: Consiste em analisar os impactos causados por uma mudança em um ou mais casos de uso
	       em outros casos de uso, e mitigar o problema, se houver.
	      
	      \item \textbf{Artefato(s) de saída}: Matriz de rastreabilidade (atualizada).
		    
	    \end{itemize}
	    
	\end{itemize}
	     
    \end{itemize}

    
    