      
   \subsection{O Modelo de Maturidade MPS.BR}
        
      O modelo de maturidade escolhido para se buscar os processos foi o Melhoria de Processo do \textit{Software} Brasileiro (MPS.BR).
      
      O MPS.BR é um programa mobilizador que visa a melhoria do processo de \textit{software} e serviços, focado principalmente
      para as micro, pequenas e médias empresas \cite{softex12}.
      Uma das bases técnicas para a definição do modelo MPS é o \textit{Capability Maturity Model Integration} (CMMI), mais especificamente
      o \textit{CMMI for Development} (CMMI-DEV\textregistered), sendo compatível com este \cite{softex12}.
      
      O modelo MPS possui quatro componentes, o Modelo de Referência MPS para \textit{Software} (MR-MPS-SW), Modelo de Referência MPS para
      Serviços (MR-MPS-SV), Método de Avaliação (MA-MPS) e Modelo de Negócio (MN-MPS), onde cada componente é descrito por um guia
      do programa MPS.BR \cite{softex12}.
      O guia utilizado para a projeto foi o do MR-MPS-SW, que traz a descrição do Modelo de Referência MPS para \textit{Software}.
      
      O MR-MPS-SW é estruturado em sete níveis de maturidade (de A a G), onde cada nível possui seus respectivos processos \cite{softex12}.
      Os processos do MR-MPS-SW são descritos por meio de propósito e resultados esperados \cite{softex12}.
      Apenas dois níveis de maturidade, o G e o D, contém atividades da Engenharia de Requisitos.
      
    \subsection{Processos e Resultados Esperados do MPS.BR atendidos pelo processo}
    
      O guia não traz as atividades e tarefas necessárias para o atendimento dos propósitos e dos resultados esperados, portanto,
      utilizamos a abordagem e o processo definidos para buscar tais atividades de modo a cumprir os seguintes processos
      e seus resultados esperados:\linebreak
      
      \noindent
      \textbf{Nível G} - Processo \textbf{Gerenciamento de Requisitos (GRE)}
	  
	  \noindent
	  Resultados esperados:
	  
	  \begin{itemize}
	    
	    \item \textbf{GRE 1}: O entendimento dos requisitos é obtido junto aos fornecedores de requisitos;
	      
	      Esse resultado é obtido com as atividades dos subprocessos "Identificar os épicos de negócio", "Levantar as features" 
	      e "Identificar casos de uso", assim como nas atividades "Fazer revisão da \textit{sprint}" 
	      e "Fazer retrospectiva da \textit{release}", onde a participação dos \textit{stakeholders} é marcante para
	      o entendimento dos requisitos.
		  
	    \item \textbf{GRE 3}: A rastreabilidade bidirecional entre os requisitos e os produtos de trabalho é estabelecida e mantida;
	      
	      Esse resultado é obtido com a rastreabilidade que é feita a partir dos artefatos de requisitos do processo,
	      onde um Tema de Investimento gera vários épicos, os épicos geram várias \textit{features} e 
	      as \textit{features} geram vários casos de uso. É possível fazer também o caminho inverso,
	      identificar a partir de um caso de uso, sua respectiva \textit{feature}, dessa \textit{feature} seu respectivo épico, e 
	      deste épico seu respectivo Tema de Investimento.
	    
	    \item \textbf{GRE 4}: Revisões em planos e produtos de trabalho do projeto são realizadas visando identificar
	      e corrigir inconsistências em relação aos requisitos;
	    
	      Esse resultado é obtido com as atividades de "Fazer revisão da \textit{sprint}" 
	      e "Fazer retrospectiva da \textit{release}", onde são validados os requisitos implementados para assegurar
	      a conformidade com os propósitos iniciais do projeto. Além da validação que é feita em diversos documentos,
	      como o Documento de Visão.
	    
	    \item \textbf{GRE 5}: Mudanças nos requisitos são gerenciadas ao longo do projeto.
	    
	      Esse resultado é obtido com as atividades dos subprocessos "Gerenciar Épicos", "Gerenciar \textit{Features}"  e
	      "Gerenciar casos de uso", e a atividade "Gerenciar \textit{sprint}", que tratam das mudanças que podem ocorrer
	      nos requisitos ao longo do projeto.
	      
	  \end{itemize}

      \noindent
      \textbf{Nível D} - Processo \textbf{Desenvolvimento de Requisitos (DRE)}
	  
	  \noindent
	  Resultados esperados:
	  
	  \begin{itemize}
	  
	  \item \textbf{DRE 1}: As necessidades, expectativas e restrições do cliente, tanto do produto quanto
	    de suas interfaces, são identificadas;
	    
	    Esse resultado é obtido através da atividade "Analisar o Negócio" do subprocesso "Identificar os épicos de negócio",
	    ao se levantar as necessidades do cliente frente ao contexto organizacional.
	  
	  \item \textbf{DRE 2}: Um conjunto definido de requisitos do cliente é especificado e priorizado a partir das
	    necessidades, expectativas e restrições identificadas;
	    
	    Esse resultado é obtido através das atividades do subprocesso "Identificar os épicos de negócio", onde os
	    requisitos de mais alto nível são elicitados (requisitos do cliente), e da atividade "Priorizar um épico", onde 
	    um épico é priorizado para a derivação das \textit{features}.
	  
	  \item \textbf{DRE 3}: Um conjunto de requisitos funcionais e não-funcionais, do produto e dos componentes do produto
	    que descrevem a solução do problema a ser resolvido, é definido e mantido a partir dos requisitos do cliente;
	    
	    Esse resultado é obtido por meio das atividades do subprocesso "Levantar \textit{features}", onde são elicitadas
	    as \textit{features} a partir dos épicos (requisitos do cliente), e pela atividade
	    "Identificar os requisitos não-funcionais", onde são levantados os requisitos não-funcionais do sistema a partir de
	    um épico priorizado.
	    
	  \item \textbf{DRE 4}: Os requisitos funcionais e não-funcionais de cada componente do produto são refinados, 
	    elaborados e alocados;
	    
	    Esse resultado é obtido através das atividades do subprocesso "Identificar casos de uso" e a atividade
	    "Especificar casos de uso", onde são elaborados
	    os artefatos de requisitos de mais baixo nível, e pelas atividades "Priorizar casos de uso" e 
	    "Planejar a \textit{sprint}", onde os requisitos são alocados em iterações. 
	    
	  \item \textbf{DRE 8}: Os requisitos são validados;
	  
	     Ambos os subprocessos "Identificar os épicos de negócio", "Levantar as features" e "Identificar casos de uso"  contêm
	     atividades de validação dos requisitos. Além disso, as atividades "Fazer retrospectiva da \textit{release}" e 
	     "Fazer revisão da \textit{sprint}" servem para validar o que foi produzido, com os \textit{stakeholders}.
	  
	  \end{itemize}
	  
	  
	  \vfill